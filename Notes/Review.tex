\documentclass{article}

\usepackage{amsfonts}
\usepackage{amsmath}
\usepackage{amsthm}
\usepackage{xcolor}
%\usepackage{nicematrix}

\setlength\parindent{0pt} %no indentation for paragraph

\newtheorem{definition}{Definition}
\newtheorem{theorem}{Theorem}
\newtheorem{proposition}{Proposition}
\newtheorem{corollary}{Corollary}[theorem] 
\newtheorem{lemma}{Lemma}

\title{Review of the dual attack on euclidean lattices}
\author{Clément Dell'Aiera - Tuong-Huy Nguyen}

\begin{document}
%%%%%%%%%%%%%%%%%
%%%%%%%%%%%%%%%%%
\maketitle
Critics of the dual attack have focused their arguments on the following points :
\begin{itemize}
\item[$\bullet$] The independence of the summands in the distiguisher.
\item[$\bullet$] The distribution of the output of a sieve.
\item[$\bullet$] The contradictory regime that follows from numerical experiment. The attack could not work for some specific choices of parameters, choices that were determined experimentally.
\end{itemize}

\subsection{Independance}

The independance heurisic refers to the following problem. Let $W\subset  \Lambda^\vee$ be a family of vectors and $t\in \mathbb R^n$, all of which are sampled from probability distributions. Even if we suppose that all vectors in $W$ are mututally independant, the random variables 
\[\cos (\langle w ,t \rangle) \text{ for }w\in W\]
are not. 

This should in principle invalidate the use of Hoeffding inequality which requires the summands to be independant. But the bound obtained does not depend on the value of $t$, which by a simple conditionning allows us to still obtain the same bound. Indeed, $\cos (\langle w ,t \rangle)$ are independant conditionally to $t$, i.e. for every family of measurable subsets $\{A_w\}_{w\in W}$ of $\mathbb R$, we have
\[ \mathbb P(\cap_{w\in W} \cos (\langle w ,t \rangle) \in A_w | t) =\prod_{w\in W} \mathbb P(\cos (\langle w ,t \rangle) \in A_w | t) .\] 
In \cite{PoulyShen}, the use of Hoeffding inequaltiy ensures that
\[\mathbb P(|g_w(t)-f(t)| > \varepsilon |t) \leq Ce^{-c\varepsilon^2}.\]
Then 
\[\begin{split}
\mathbb P(|g_w(t)-f(t)| > \varepsilon ) & =\int \mathbb P(|g_w(t)-f(t)| > \varepsilon |t = t')\mathbb P_t(dt') \\
		& \leq Ce^{-c\varepsilon^2}\\
\end{split}\]
Thus, we see that $g_w(t)$ concentrated around its mean value $f(t)$, even in the independance heuristic.

\subsection{Output of a sieve}

The distribution of vectors in $W$ also influence the validity of the distinguisher. Indeed, the analysis carried in \cite{PoulyShen} relies on the assumption that the element of $W$ are sampled independantly from $D_{\Lambda^\vee ,s^{-1}}$. In that case, the distinguisher can be determined via an application of the Poisson summation formula. We have that 
\[f(t) = \mathbb E[g_W(t)|t ] = \frac{\rho_s(\Lambda + t)}{\rho_s(\Lambda)}.\] 
Using inequalities from the ,

details about isotrpoy 


\bibliographystyle{plain}
\bibliography{biblio}
\end{document}
