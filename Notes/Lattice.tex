\documentclass{article}

\usepackage{amsfonts}
\usepackage{amsmath}
\usepackage{amsthm}
\usepackage{xcolor}
%\usepackage{nicematrix}

\setlength\parindent{0pt} %no indentation for paragraph

\newtheorem{definition}{Definition}
\newtheorem{theorem}{Theorem}
\newtheorem{proposition}{Proposition}
\newtheorem{corollary}{Corollary}[theorem] 
\newtheorem{lemma}{Lemma}

\title{Lattice based cryptography}
\author{Clément Dell'Aiera}

\begin{document}
%%%%%%%%%%%%%%%%%
%%%%%%%%%%%%%%%%%
\maketitle

%%%%%%%%%%%%%%%%%%%%%%%%%%%%
\newpage	%%%%%%%%%%%%
%%%%%%%%%%%%%%%%%%%%%%%%%%%%
\section{Lattices}

Every locally compact group admits a left-invariant regular Borel measure, and all of these are proportional. These are called Haar measures. A lattice in a l.c. group is a closed subgroup of finite co-volume\footnote{i.e. the pushed-forward on the quotient of any Haar measure is finite.}. Such groups are always discrete. 

\subsection{Euclidean lattices}
Every lattice of $\mathbb R^n$ is a discrete subgroup, and is of the form 
\[\Lambda_B = \{ Bv \ : \ v\in \mathbb Z^n\} \]
for some invertible matrix $B\in GL(n,\mathbb R)$, called a basis. 

The dual $\Lambda^\vee$ is the euclidean lattice
\[\{ w \in V \ : \  \langle w , \Lambda \rangle \subset \mathbb Z \}.\]
The map 
$$w \in \Lambda^\vee \mapsto \left(t\in V \mapsto \exp(2i\pi \langle w,t \rangle)\right)$$ 
induces an isomorphism 
\[\Lambda^\vee \cong \widehat{V/\Lambda}.\] 
A consequence of which is that every (for instance square-integrable) $\Lambda$-periodic function on $V$ can be expressed as the (hilbertian) weighted sum of characters. This\footnote{I forgot some normalizing constant.} is the isomorphism given by the Fourier transform $\mathcal F : L^2(V/\Lambda ) \rightarrow \ell^2(\Lambda^\vee) $
\[\mathcal F(\varphi)(w) = \int_V \exp(-2i\pi \langle w,t \rangle) \varphi(t)dt. \]
\[\mathcal F^{*}(\psi)(t) = \sum_{w\in\Lambda^\vee} \psi_w \exp(2i\pi \langle w,t \rangle) . \]

\subsection{Poisson formula}
The Fourier transform
\[\mathcal F(\varphi ) (\xi ) = \int_V \varphi(v)e^{-2i\pi \xi(x)} dm(v) \quad\forall \xi \in V^\vee.\]
induces an isomorphism of topological vector spaces
\[\mathcal F : \mathcal S(V) \rightarrow \mathcal S(V^\vee)\]
between Schwartz spaces of $V$ and its dual (as a vector space), which extends to an isomorphism
\[\mathcal S'(V) \rightarrow \mathcal S'(V^\vee)\]
of tempered distributions.
The Poisson formula is the statement that for every Schwartz class function $\varphi\in\mathcal S(V)$, 
\[ \varphi (\Lambda + t) = covol(\Lambda)^{-1}(\mathcal F(\varphi)e^{2i\pi \langle \bullet , t\rangle }) (\Lambda^\vee).\]
It follows from the fact that the counting measures on $\Lambda$ and $\Lambda^\vee$ are related by 
\[\mathcal F(\sum_{v\in \Lambda} \delta_{v} ) = covol(\Lambda)^{-1}\sum_{w\in \Lambda^\vee} \delta_w .\]
For the gaussian function $\rho_s(t) = \exp(-s\pi\|t\|^2)$, 
\[\mathcal F(\rho_s)(\xi)= s^{-n/2}\exp(-s^{-1}\pi\|\xi\|^2 ),\]
which yields the classical identity\footnote{$\Theta_{\Lambda}(s) = \rho_s(\Lambda)$.} for thêta functions
\[ \Theta_{\Lambda}(s) = covol(\Lambda)^{-1}s^{-n/2}\Theta_{\Lambda^\vee}(s^{-1}).\]

%%%%%%%%%%%%%%%%%%%%%%%%%%%%%%%%%%%%%%%%%%%%%%%%%%%%%%%%%
\subsection{Kernels and Fourier transforms on lattices} %
%%%%%%%%%%%%%%%%%%%%%%%%%%%%%%%%%%%%%%%%%%%%%%%%%%%%%%%%%

%Another version of the Fourier transform acts on $\ell^1$ functions. 
For every abelian discrete group $G$, let 
\[\ell^1(G) = \{ f : G \rightarrow \mathbb C \ :\ \sum_{g\in G}|f(g)| < \infty \}.\]
Endowed with the convolution product and the $\ell^1$-norm, it is a commutative involutive complex Banach algebra. Its Pontryagin dual $\hat G$ is the group of homomorphisms $\chi : G\rightarrow \mathbb S^1$, i.e.
\[\chi(e_G ) = 1 \text{ and } \chi(st) = \chi(s)\chi(t) \quad \forall s,t \in G,\]
endowed with the topology of uniform convergence on compact subsets. As $G$ is discrete, $\hat G$ is compact. Let us denote by CP$(\hat G)$ the space of continuous normalized functions of positive types on $\hat G$, i.e. functions $f : \hat G \rightarrow \mathbb C$ such that $f(0) = 1$ and for every real numbers $a_1,\ldots , a_n$ and every elements $g_1,\ldots , g_n$ in $G$, the following inequality is satisfied :
\[\sum_{i=1}^n a_i a_j f(g_ig_j^{-1})\geq 0.\] 
 
The Gelfand transform on $\ell^1(G)$ coincides with the Fourier transform on $G$, and yields an injective continuous morphism $\mathfrak F : \ell^1(G)\hookrightarrow C(\hat G)$. In the case of (dual of) lattices, we get
\[\mathfrak F : \left\{ \begin{array}{rcl}
\ell^1(\Lambda^\vee) & \rightarrow &  C(\mathbb R^n / \Lambda) \\
\eta & \mapsto & \hat \eta : t \mapsto \sum_{w\in \Lambda^\vee} \eta(w) e^{2i\pi \langle w , t\rangle} \\ 
\end{array}\right.\]

The restriction of $\mathfrak F$ to the subset of probability measures
\[\text{Prob}(\Lambda^\vee) =\{\eta\in\ell^1(\Lambda^\vee) \ :\ \eta(w)\geq 0 \text{ and } \sum_{w\in \Lambda^\vee} \eta(w) = 1\}  \]
is the characteristic function, i.e. for every $\eta\in\text{Prob} (\Lambda^\vee)$, if $W$ is a random variable distributed according to $\eta$, 
\[\mathfrak F(\eta)(t) = \mathbb E[e^{2i\pi \langle W , t\rangle } ]\quad\forall t\in \mathbb R^n.\] 
By Bochner theorem, $\mathfrak F(\text{Prob} (\Lambda^\vee))$ is the set of normalized continuous kernels of positive types on $\mathbb R^n / \Lambda$. Thus we get a continuous isomorphism 
\[\mathfrak F : \text{Prob}(\Lambda^\vee ) \rightarrow \text{CP}(\mathbb R^n / \Lambda).\]

We can summarized the discussion above by the following theorem.

\begin{theorem}[Bochner theorem for euclidean lattices]\label{BochnerLattices}
For every continuous $\Lambda$-periodic function $f : \mathbb R^n \rightarrow \mathbb C$ of positive type such that $f(0) = 1$, there exists a probability measure $\eta\in \text{Prob} (\Lambda^\vee)$ such that 
\[f(t) =\mathbb E[e^{2i\pi \langle W , t\rangle } ]\quad\forall t\in \mathbb R^n ,\]
where $W$ is a random variable with law $\eta$.
\end{theorem}


%%%%%%%%%%%%%%%%%%%%%%%%%%%%%%%%%%
\subsection{Congruence lattices} %
%%%%%%%%%%%%%%%%%%%%%%%%%%%%%%%%%%
A congruence lattice\footnote{The accepted terminology in the computer science community is $q$-ary.} is a lattice $\Lambda$ of $\mathbb R^n$ such that 
\[q\mathbb Z^n < \Lambda < \mathbb Z^n,\]
for some positive integer $q$, called the modulus. When useful, we will use the term $q$-congruence lattice if we want to make the modulus explicit.
 
As inclusion of lattices $\Lambda_{B_0} < \Lambda_{B_1}$ translates as a relation
\[\exists g \in  GL(n,\mathbb R) \cap M_n(\mathbb Z) , B_0 = B_1g , \] 
these inclusions entails basis of congruence lattices are matrices in 
\[\{B\in M(n,\mathbb Z) : \exists g\in GL(n,\mathbb Z[\frac{1}{q}] ) \ : \ Bg = gB = qI_n \}\]
%$M(n,q\mathbb Z)\cap GL(n,\mathbb R) \subset GL(n,\mathbb Z[\frac{1}{q}])$, i.e.
%\[\mathcal L_{n,q} \cong \left(M(n,q\mathbb Z)\cap GL(n,\mathbb Z[ \frac{1}{q} ])\right) / GL(n,\mathbb Z)\] 
   
\subsubsection{}

For $A\in (\mathbb Z/q\mathbb Z)^{n\times k}$, we define the lattices
\[\mathcal L(A) = \{x\in\mathbb Z^{n} \ : \ \exists y\in\mathbb Z^k \text{s.t. } x = Ay \pmod{q} \}\]
and
\[\mathcal L(A)^\perp = \{x\in\mathbb Z^{n} \ :\ A^{T}x = 0 \pmod{q} \}.\]
These lattices are $q$-congruent. Moreover $q\mathcal L(A)^\vee = \mathcal L(A)^\perp$.   

Removing any set of columns from $A$ such that the obtained matrix has same range doesn't change $\mathcal L(A)$. We can thus suppose that $m>n$ and that $A$ is a maximal rank, rank$(A)=n$. Up to a change of basis, $A$ can be thus be supposed of the type 
\[\begin{pmatrix}
I_k \\
A_0
\end{pmatrix}\]
with $A_0\in (\mathbb Z/q\mathbb Z)^{m\times n}$. In that case a basis for $\mathcal L(A)$ is 
\[
\begin{pmatrix}
  I_m & A_0 \\ 0 & qI_k 
\end{pmatrix}
\]
and a basis for $\mathcal L(A)^\perp$ is 
\[
\begin{pmatrix}
  q I_m & 0 \\  - A_0 & I_k \\
\end{pmatrix},
\]
which, once multiplied by $q^{-1}$, gives a basis of $\mathcal L(A)^\vee$. We also have covol$(\mathcal L(A))=q^{k}$, covol$(\mathcal L(A)^\perp)=q^{m}$, and of course covol$(\mathcal L(A))^\vee=q^{-k}$ as expected.

%%%%%%%%%%%%%%%%%%%%%%%%%%%%%%%%%%%%%%%%
\subsection{Random euclidean lattices} %
%%%%%%%%%%%%%%%%%%%%%%%%%%%%%%%%%%%%%%%%
The map that sends a basis to the generated lattice yields a measurable model 
\[\mathcal L_n \cong  GL(n,\mathbb R) / GL(n,\mathbb Z) \]
for the space of lattices. Since $GL(n,\mathbb Z)$ is a lattice in $GL(n,\mathbb R)$, $\mathcal L_n$ admits a unique invariant probability measure $\mu_n$. In a similar manner, lattices of covolume $\rho$ admit an invariant probability measure from the model 
\[\mathcal L_{n,\rho} \cong \mathcal L_{n,1}\cong SL(n,\mathbb R) / SL(n,\mathbb Z)\]
These measures are called the Haar-Siegel probability measures in the litterature.


% Ajouter les résultats sur l'heuristic gaussienne vu comme une approximation de E[f(L)]  = \int f d\mu [section 2.5 de Dadush Regev]
% d'où E[ | L\{0} \cap A|] = Vol(A)
% Goldstein Mayer 2003 : la mesure uniforme (en fait ils considèrent la mesure uniforme sur l'ensemble fini des réseaux entiers de covolume fixé égal à N) sur les réseaux de congruence converge en un sens fort vers la mesure \mu . Les réseaux de congruence ne sont pas exactements ceux là, leur volume est un diviseur de q^n, de plus un réseau entier de covolume N ne contient pas nécessairement NZZ^n.
% Pouly Shen donne dans D3 et D4 une version quantitative de l'heuristique gaussienne pour les réseaux de congruence. 
\subsubsection{}
Let $\Gamma_0 , \Gamma_1$ be two lattices in a l.c. group $G$ such that $\Gamma_0$ is of finite index in $\Gamma_1$, then 
\[ \text{covol } \Gamma_0  = [\Gamma_1 : \Gamma_0 ] \ \text{covol } \Gamma_1 .\]
As a consequence, $\text{covol }\Lambda $ is equal to the index $[\mathbb Z^n : \Lambda]$ and divides $q^n$.

Let $\Lambda$ be an euclidean lattice, and $F$ be a finite group of cardinal $k$. Let us define $\mathcal L_{n , F}(\Lambda ) $ be the set of lattices $\Lambda'$ containing $\Lambda$ and satifying that $\Lambda' / \Lambda $ is isomorphic to $F$. This set is finite and it follows from \cite{ordentlich2022new} (proposition 2.1) that
\[\int \varphi d\mu_{n,k^{-1}} = \int_{\mathcal L_n} \frac{1}{k}\sum_{\Lambda'\in \mathcal L_{n , F}(\Lambda )} \varphi(\Lambda') d\mu_{n}(\Lambda)  \quad \forall \varphi \in C_c(\mathcal L_{n,k^{-1}} ) \]
As a consequence, the uniform probability measure on $q$-congruence lattices is the law $\Lambda'$ obtained by drawing $\Lambda$ according to $\mu_{n,1}$, and then drawing $\Lambda'$ in $\mathcal L_{n,\mathbb Z^n /q\mathbb Z^n}(\Lambda)$ uniformly, conditionally to the event $\{\Lambda = q\mathbb Z^n\}$.

\subsubsection{Equidistribution of Hecke points}
Let $N$ be a positive integer. Let us denote by $\mathfrak H_N$ the space of integer lattices of covolume $N$. It is a finite set. Renormalizing by $N^{-\frac{1}{n}}$ gives a map $\mathfrak H_N \rightarrow \mathcal L_{n,1}$: let us denote by $\mathfrak h_{N}$ its push-forward of the uniform probability measure on $\mathfrak H_N$. 

\begin{theorem}[Theorem 2.2 of \cite{goldstein2003equidistribution}]
For every $f\in C_c(\mathcal L_{n,1})$
\[\lim_{N\rightarrow +\infty} \int_{\mathcal L_{n,1}} f d\mathfrak h_{N} = \int_{\mathcal L_{n,1}} fd\mu_{n,1}\]
\end{theorem}

This result allows to approximate the Haar-Siegel probability measure by the measure obtained by fixing a big integer $N$, drawing uniformly a random element of the finite set $\mathfrak H_N$, then renormalizing by $N^{-\frac{1}{n}}$.

\subsubsection{Siegel's mean theorem}

Siegel proved the following result. Generalizations are te be found elsewhere, see Schmidt, or Rogers, Macbeath  \cite{macbeath1958siegel}.
Rogers paper \cite{rogers1955mean} contains a mistake for $n=2$ that was latter fixed in \cite{schmidt1960metrical}.

\begin{theorem}[Siegel's mean value theorem \cite{siegel1945mean}]
Let $L$ be a random lattice distributed according to the normalized Haar measure $\mu_{n,1}$. For every Lebesgue-integrable function $f\in L^1(\mathbb R^{n},\lambda)$,
\[\mathbb E[\sum_{v\in L}f(v)] = \int_{\mathbb R^n}f(x) d\lambda(x).\]
\end{theorem}
%%%%%%%%%%%%%%%%%%%%%%%%%
%%%%%%%%%%%%%%%%%%%%%%%%%
\bibliographystyle{plain}
\bibliography{biblio}
\end{document}