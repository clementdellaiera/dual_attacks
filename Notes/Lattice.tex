\documentclass{article}

\usepackage{amsfonts}
\usepackage{amsmath}
\usepackage{amsthm}

%\usepackage{nicematrix}

\setlength\parindent{0pt} %no indentation for paragraph

\newtheorem{definition}{Definition}
\newtheorem{theorem}{Theorem}
\newtheorem{proposition}{Proposition}
\newtheorem{lemma}{Lemma}

\title{Lattice based cryptography}
\author{Clément Dell'Aiera}

\begin{document}
%%%%%%%%%%%%%%%%%
%%%%%%%%%%%%%%%%%
\maketitle

%%%%%%%%%%%%%%%%%%%%%%%%%%%%
\newpage	%%%%%%%%%%%%
%%%%%%%%%%%%%%%%%%%%%%%%%%%%
\section{Lattices}

Every locally compact group admits a left-invariant regular Borel measure, and all of these are proportional. These are called Haar measures. A lattice in a l.c. group is a closed subgroup of finite co-volume\footnote{i.e. the pushed-forward on the quotient of any Haar measure is finite.}. Such groups are always discrete. 

\subsection{Euclidean lattices}
Every lattice of $\mathbb R^n$ is a discrete subgroup, and is of the form 
\[\Lambda_B = \{ Bv \ : \ v\in \mathbb Z^n\} \]
for some invertible matrix $B\in GL(n,\mathbb R)$, called a basis. 

The dual $\Lambda^\vee$ is the euclidean lattice
\[\{ w \in V \ : \  \langle w , \Lambda \rangle \subset \mathbb Z \}.\]
The map 
$$w \in \Lambda^\vee \mapsto \left(t\in V \mapsto \exp(2i\pi \langle w,t \rangle)\right)$$ 
induces an isomorphism 
\[\Lambda^\vee \cong \widehat{V/\Lambda}.\] 
A consequence of which is that every (for instance square-integrable) $\Lambda$-periodic function on $V$ can be expressed as the (hilbertian) weighted sum of characters. This\footnote{I forgot some normalizing constant.} is the isomorphism given by the Fourier transform $\mathcal F : L^2(V/\Lambda ) \rightarrow \ell^2(\Lambda^\vee) $
\[\mathcal F(\varphi)(w) = \int_V \exp(-2i\pi \langle w,t \rangle) \varphi(t)dt. \]
\[\mathcal F^{*}(\psi)(t) = \sum_{w\in\Lambda^\vee} \psi_w \exp(2i\pi \langle w,t \rangle) . \]

\subsection{Poisson formula}
The Fourier transform
\[\mathcal F(\varphi ) (\xi ) = \int_V \varphi(v)e^{-2i\pi \xi(x)} dm(v) \quad\forall \xi \in V^\vee.\]
induces an isomorphism of topological vector spaces
\[\mathcal F : \mathcal S(V) \rightarrow \mathcal S(V^\vee)\]
between Schwartz spaces of $V$ and its dual (as a vector space), which extends to an isomorphism
\[\mathcal S'(V) \rightarrow \mathcal S'(V^\vee)\]
of tempered distributions.
The Poisson formula is the statement that for every Schwartz class function $\varphi\in\mathcal S(V)$, 
\[ \varphi (\Lambda + t) = covol(\Lambda)^{-1}(\mathcal F(\varphi)e^{2i\pi \langle \bullet , t\rangle }) (\Lambda^\vee).\]
It follows from the fact that the counting measures on $\Lambda$ and $\Lambda^\vee$ are related by 
\[\mathcal F(\sum_{v\in \Lambda} \delta_{v} ) = covol(\Lambda)^{-1}\sum_{w\in \Lambda^\vee} \delta_w .\]
For the gaussian function $\rho_s(t) = \exp(-s\pi\|t\|^2)$, 
\[\mathcal F(\rho_s)(\xi)= s^{-n/2}\exp(-s^{-1}\pi\|\xi\|^2 ),\]
which yields the classical identity\footnote{$\Theta_{\Lambda}(s) = \rho_s(\Lambda)$.} for thêta functions
\[ \Theta_{\Lambda}(s) = covol(\Lambda)^{-1}s^{-n/2}\Theta_{\Lambda^\vee}(s^{-1}).\]

%%%%%%%%%%%%%%%%%%%%%%%%%%%%%%%%%%%%%%%%%%%%%%%%%%%%%%%%%
\subsection{Kernels and Fourier transforms on lattices} %
%%%%%%%%%%%%%%%%%%%%%%%%%%%%%%%%%%%%%%%%%%%%%%%%%%%%%%%%%

%Another version of the Fourier transform acts on $\ell^1$ functions. 
For every abelian discrete group $G$, let 
\[\ell^1(G) = \{ f : G \rightarrow \mathbb C \ :\ \sum_{g\in G}|f(g)| < \infty \}.\]
Endowed with the convolution product and the $\ell^1$-norm, it is a commutative involutive complex Banach algebra. Its Pontryagin dual $\hat G$ is the group of homomorphisms $\chi : G\rightarrow \mathbb S^1$, i.e.
\[\chi(e_G ) = 1 \text{ and } \chi(st) = \chi(s)\chi(t) \quad \forall s,t \in G,\]
endowed with the topology of uniform convergence on compact subsets. As $G$ is discrete, $\hat G$ is compact. Let us denote by CP$(\hat G)$ the space of continuous normalized functions of positive types on $\hat G$, i.e. functions $f : \hat G \rightarrow \mathbb C$ such that $f(0) = 1$ and for every real numbers $a_1,\ldots , a_n$ and every elements $g_1,\ldots , g_n$ in $G$, the following inequality is satisfied :
\[\sum_{i=1}^n a_i a_j f(g_ig_j^{-1})\geq 0.\] 
 
The Gelfand transform on $\ell^1(G)$ coincides with the Fourier transform on $G$, and yields an injective continuous morphism $\mathfrak F : \ell^1(G)\hookrightarrow C(\hat G)$. In the case of (dual of) lattices, we get
\[\mathfrak F : \left\{ \begin{array}{rcl}
\ell^1(\Lambda^\vee) & \rightarrow &  C(\mathbb R^n / \Lambda) \\
\eta & \mapsto & \hat \eta : t \mapsto \sum_{w\in \Lambda^\vee} \eta(w) e^{2i\pi \langle w , t\rangle} \\ 
\end{array}\right.\]

The restriction of $\mathfrak F$ to the subset of probability measures
\[\text{Prob}(\Lambda^\vee) =\{\eta\in\ell^1(\Lambda^\vee) \ :\ \eta(w)\geq 0 \text{ and } \sum_{w\in \Lambda^\vee} \eta(w) = 1\}  \]
is the characteristic function, i.e. for every $\eta\in\text{Prob} (\Lambda^\vee)$, if $W$ is a random variable distributed according to $\eta$, 
\[\mathfrak F(\eta)(t) = \mathbb E[e^{2i\pi \langle W , t\rangle } ]\quad\forall t\in \mathbb R^n.\] 
By Bochner theorem, $\mathfrak F(\text{Prob} (\Lambda^\vee))$ is the set of normalized continuous kernels of positive types on $\mathbb R^n / \Lambda$. Thus we get a continuous isomorphism 
\[\mathfrak F : \text{Prob}(\Lambda^\vee ) \rightarrow \text{CP}(\mathbb R^n / \Lambda).\]

We can summarized the discussion above by the following theorem.

\begin{theorem}[Bochner theorem for euclidean lattices]\label{BochnerLattices}
For every continuous $\Lambda$-periodic function $f : \mathbb R^n \rightarrow \mathbb C$ of positive type such that $f(0) = 1$, there exists a probability measure $\eta\in \text{Prob} (\Lambda^\vee)$ such that 
\[f(t) =\mathbb E[e^{2i\pi \langle W , t\rangle } ]\quad\forall t\in \mathbb R^n ,\]
where $W$ is a random variable with law $\eta$.
\end{theorem}

The function 
\[f(t) = \frac{\rho_s(t+\Lambda)}{\rho_s(\Lambda)}\]
is such a kernel, and is thus in the image of the Fourier transform. Using the Poisson formula applied to the discrete gaussian function, we get 
%We can actually determine ita analytically using  the Fourier transform of the gaussian function and the Poisson formula : $f = \mathcal F(\eta) $ for 
%\[\eta(w) = \frac{\rho_{\frac{1}{s}} (\|w\|^2)}{\rho_s(\Lambda) \det(\Lambda)} \quad\forall w\in \Lambda^\vee.\] 
\[\begin{split} 
f(t) & = \frac{ \rho_s (\Lambda + t)}{\rho_s (\Lambda )} \\ 
	& = \frac{ ( \rho_{s^{-1}} e^{2i\pi \langle \bullet , t\rangle }) (\Lambda^\vee) }{ \rho_{s^{-1}}(\Lambda^\vee)} \\
	& = \rho_{s^{-1}}(\Lambda^\vee)^{-1}\sum_{w\in \Lambda^\vee} \rho_{s^{-1}}(w) \cos (2\pi \langle w,t\rangle ) .
\end{split}\]
This ensures that if $W$ is a random variable following $D_{\Lambda^\vee, s^{-1}}$, then 
\[f(t) = \mathbb E[\cos (2\pi \langle W , t \rangle ) ] \quad \forall t\in \mathbb R^n.\] 

Let $W_1,\ldots , W_N$ be i.i.d random variables with law $D_{\Lambda^\vee, s^{-1}}$. The random variables $\cos (2\pi \langle W , t \rangle ) $ are bounded. 
Let us denote $\frac{1}{N}\sum_{k=1}^N \cos (2\pi \langle W_k , t \rangle )$ by $g_N(t)$. Then , by Chebyshev inequality,
\[\mathbb P( |g_N(t) - f(t) |\geq \varepsilon ) \leq  2 \exp( - \frac{2\varepsilon^2}{4N}) \quad \forall \varepsilon >0.\]

The provable regime of Pouly-Shen follows from the properties of $\frac{ \rho_s (\Lambda + t)}{\rho_s (\Lambda )}$ as a distinguisher, which highly depends on $s$. This analysis breaks down if we consider $W_1,\ldots , W_N$ sampled from a sieve (and not sampled as independent draws of $D_{\Lambda,s^{-1}}$). 

However, most papers (including \cite{ducas2023accurate}) assume that the output of a sieve is a sample of independent variables distributed as a uniform random variable on a Euclidean ball. In that case, the distinguisher can be computed as $f(t) = \mathbb E[e^{2i\pi\langle W , t \rangle} )]$ for $W\sim \mathcal U(B_n(r_{\text{sat}} \mathfrak{gh}_n))$ and approximated with a sample of size $N = \frac{1}{2}f_{\text{sat}}r_{\text{sat}} $ (see heuristic 2 of \cite{ducas2023accurate}, page 7). It remains to study if the behavious of $f(t)$ with respect to the distance. One reason for optimism is that gaussians in high dimensions are concentrated on spheres, thus I would expect the behaviours to be close.    

\subsubsection{Analogy with codes}

For $\mathbb F_q$, where $q = p^\ell$, recall (see corollary 7.1.3 of \cite{scarabotti2018discrete}) that $\mathbb F_q$ and its dual are isomorphic as groups via the following isomorphism 
\[\left\{ \begin{array}{rcl}
\mathbb F_q 	& \rightarrow 	& \hat {\mathbb F_q} \\
x	& \mapsto 	& \left( y\mapsto \exp (\frac{2k\pi }{p}\text{Tr}_{\mathbb F_q / \mathbb F_p}(xy)) \right)\\
\end{array}\right.\]
The Fourier inversion formula also holds for every function $f : \mathbb F_q \rightarrow \mathbb C$ :
\[f(x) = \frac{1}{q} \sum_{\chi \in \hat{\mathbb F_q}} \hat f(\chi) \chi \]
For functions on $\mathbb F_q^n$ that are $C$-periodic, we are looking at the Fourier transform on the finite abelian group $\mathbb F_q / C$, whose dual is described by
\[ \chi_x(y) = \exp (\frac{2k\pi }{p}(x,y)_{\mathbb F_p} )\]
where $y \in C^\perp$, for 
\[C^{\perp} = \{y\in \mathbb F_q^n \ : \ (x,y)_{\mathbb F_p}= 0   \}.\]
and 
\[(x,y)_{\mathbb F_p} = \sum_{i=1}^n \text{Tr}_{\mathbb F_q / \mathbb F_p}(x_iy_i) \]
Several papers notice an analogy with dual attacks for linear codes. The latter are still abelian groups. Since they are finite, all functions are continuous. Moreover, $\text{CP}(\mathbb F_q^n / C)$ is the set of positive definite matrices indexed by $\mathbb F_q^n / C$. In this case the Fourier isomorphism of theorem \ref{BochnerLattices} takes the following form.

\begin{theorem}
For every $C$-periodic function $f : \mathbb F_q^n \rightarrow \mathbb C$ of positive type such that $f(0) = 1$, there exists a probability measure $\eta\in \text{Prob} (C^\vee)$ such that 
\[f(t) =\mathbb E[e^{\frac{2i\pi}{p} ( W , t)_{\mathbb F_p} } ]\quad\forall t\in \mathbb F_q^n ,\]
where $W$ is a random variable with law $\eta$.
\end{theorem}

Let us denote by $w(x)$ the Hamming weight of a vector $x\in\mathbb F^n_q$
\[w(x) = \sum_{i=1}^n 1_{(x_i\neq 0)}\]
and $d$ the corresponding distance.

\begin{lemma}
The Hamming distance is an invariant condionnally negative type kernel on $\mathbb F^n_q$.
\end{lemma}

\begin{proof}
The following function \[\Phi : \left\{ \begin{array}{rcl}
\mathbb F^n_q 	& \rightarrow 	& \mathbb R^{n\times q} \\
(x_i)_{i=1,n}	& \mapsto 	& [1_{(x_i = a )}]_{i=1,n , a\in\mathbb F_q}\\
\end{array}\right.\]
satisfies $\|\Phi(x)-\Phi(y)\|^2 = d(x,y) = w(x-y)$, hence the result.
\end{proof}
As a corollary, for every $t>0$ and $x\in C$, there exists a random variable $W$ with values in $C^{\perp}$ such that
\[\mathbb E[e^{\frac{2i\pi}{p} ( W , x)_{\mathbb F_p}}] = \Theta_{C}(t)^{-1}\sum_{c\in C} \exp(-tw(x-c))\]
with $\Theta_{C}(t) = \sum_{c\in C} \exp(-tw(c))$.

The summation Poisson formula takes the following form.
\begin{theorem}[Poisson formula for linear codes]
For every function $\varphi : \mathbb F_q^n \rightarrow \mathbb C$, the following equality 
\[\sum_{c\in C} \varphi(x-c) = \sum_{w\in C^\perp} \hat\varphi(w)e^{2i\pi \frac{\langle w , x\rangle }{q}}\quad\forall x\in \mathbb F_q^n\]
is satisfied.
\end{theorem}
We will use the notation 
\[\varphi (C - x) = \left( e^{-2i\pi \frac{\langle w , \bullet\rangle }{q}}\hat \varphi \right) (C^\perp) \]

%%%%%%%%%%%%%%%%%%%%%%%%%%%%%%%%%%
\subsection{Congruence lattices} %
%%%%%%%%%%%%%%%%%%%%%%%%%%%%%%%%%%
A congruence lattice\footnote{The accepted terminology in the computer science community is $q$-ary.} is a lattice $\Lambda$ of $\mathbb R^n$ such that 
\[q\mathbb Z^n < \Lambda < \mathbb Z^n,\]
for some positive integer $q$, called the modulus. When useful, we will use the term $q$-congruence lattice if we want to make the modulus explicit.
 
As inclusion of lattices $\Lambda_{B_0} < \Lambda_{B_1}$ translates as a relation
\[\exists g \in  GL(n,\mathbb R) \cap M_n(\mathbb Z) , B_0 = B_1g , \] 
these inclusions entails basis of congruence lattices are matrices in 
\[\{B\in M(n,\mathbb Z) : \exists g\in GL(n,\mathbb Z[\frac{1}{q}] ) \ : \ Bg = gB = qI_n \}\]
%$M(n,q\mathbb Z)\cap GL(n,\mathbb R) \subset GL(n,\mathbb Z[\frac{1}{q}])$, i.e.
%\[\mathcal L_{n,q} \cong \left(M(n,q\mathbb Z)\cap GL(n,\mathbb Z[ \frac{1}{q} ])\right) / GL(n,\mathbb Z)\] 
   
\subsubsection{}

For $A\in (\mathbb Z/q\mathbb Z)^{n\times k}$, we define the lattices
\[\mathcal L(A) = \{x\in\mathbb Z^{n} \ : \ \exists y\in\mathbb Z^k \text{s.t. } x = Ay \pmod{q} \}\]
and
\[\mathcal L(A)^\perp = \{x\in\mathbb Z^{n} \ :\ A^{T}x = 0 \pmod{q} \}.\]
These lattices are $q$-congruent. Moreover $q\mathcal L(A)^\vee = \mathcal L(A)^\perp$.   

Removing any set of columns from $A$ such that the obtained matrix has same range doesn't change $\mathcal L(A)$. We can thus suppose that $m>n$ and that $A$ is a maximal rank, rank$(A)=n$. Up to a change of basis, $A$ can be thus be supposed of the type 
\[\begin{pmatrix}
I_k \\
A_0
\end{pmatrix}\]
with $A_0\in (\mathbb Z/q\mathbb Z)^{m\times n}$. In that case a basis for $\mathcal L(A)$ is 
\[
\begin{pmatrix}
  I_m & A_0 \\ 0 & qI_k 
\end{pmatrix}
\]
and a basis for $\mathcal L(A)^\perp$ is 
\[
\begin{pmatrix}
  q I_m & 0 \\  - A_0 & I_k \\
\end{pmatrix},
\]
which, once multiplied by $q^{-1}$, gives a basis of $\mathcal L(A)^\vee$. We also have covol$(\mathcal L(A))=q^{k}$, covol$(\mathcal L(A)^\perp)=q^{m}$, and of course covol$(\mathcal L(A))^\vee=q^{-k}$ as expected.


\subsection{Random euclidean lattices}
The map that sends a basis to the generated lattice yields a measurable model 
\[\mathcal L_n \cong  GL(n,\mathbb R) / GL(n,\mathbb Z) \]
for the space of lattices. Since $GL(n,\mathbb Z)$ is a lattice in $GL(n,\mathbb R)$, $\mathcal L_n$ admits a unique invariant probability measure $\mu_n$. In a similar manner, lattices of covolume $\rho$ admit an invariant probability measure from the model 
\[\mathcal L_{n,\rho} \cong \mathcal L_{n,1}\cong SL(n,\mathbb R) / SL(n,\mathbb Z)\]
These measures are called the Haar-Siegel probability measures in the litterature.


% Ajouter les résultats sur l'heuristic gaussienne vu comme une approximation de E[f(L)]  = \int f d\mu [section 2.5 de Dadush Regev]
% d'où E[ | L\{0} \cap A|] = Vol(A)
% Goldstein Mayer 2003 : la mesure uniforme (en fait ils considèrent la mesure uniforme sur l'ensemble fini des réseaux entiers de covolume fixé égal à N) sur les réseaux de congruence converge en un sens fort vers la mesure \mu . Les réseaux de congruence ne sont pas exactements ceux là, leur volume est un diviseur de q^n, de plus un réseau entier de covolume N ne contient pas nécessairement NZZ^n.
% Pouly Shen donne dans D3 et D4 une version quantitative de l'heuristique gaussienne pour les réseaux de congruence. 
\subsubsection{}
Let $\Gamma_0 , \Gamma_1$ be two lattices in a l.c. group $G$ such that $\Gamma_0$ is of finite index in $\Gamma_1$, then 
\[ \text{covol } \Gamma_0  = [\Gamma_1 : \Gamma_0 ] \ \text{covol } \Gamma_1 .\]
As a consequence, $\text{covol }\Lambda $ is equal to the index $[\mathbb Z^n : \Lambda]$ and divides $q^n$.

Let $\Lambda$ be an euclidean lattice, and $F$ be a finite group of cardinal $k$. Let us define $\mathcal L_{n , F}(\Lambda ) $ be the set of lattices $\Lambda'$ containing $\Lambda$ and satifying that $\Lambda' / \Lambda $ is isomorphic to $F$. This set is finite and it follows from \cite{ordentlich2022new} (proposition 2.1) that
\[\int \varphi d\mu_{n,k^{-1}} = \int_{\mathcal L_n} \frac{1}{k}\sum_{\Lambda'\in \mathcal L_{n , F}(\Lambda )} \varphi(\Lambda') d\mu_{n}(\Lambda)  \quad \forall \varphi \in C_c(\mathcal L_{n,k^{-1}} ) \]
As a consequence, the uniform probability measure on $q$-congruence lattices is the law $\Lambda'$ obtained by drawing $\Lambda$ according to $\mu_{n,1}$, and then drawing $\Lambda'$ in $\mathcal L_{n,\mathbb Z^n /q\mathbb Z^n}(\Lambda)$ uniformly, conditionally to the event $\{\Lambda = q\mathbb Z^n\}$.

\subsubsection{Equidistribution of Hecke points}
Let $N$ be a positive integer. Let us denote by $\mathfrak H_N$ the space of integer lattices of covolume $N$. It is a finite set. Renormalizing by $N^{-\frac{1}{n}}$ gives a map $\mathfrak H_N \rightarrow \mathcal L_{n,1}$: let us denote by $\mathfrak h_{N}$ its push-forward of the uniform probability measure on $\mathfrak H_N$. 

\begin{theorem}[Theorem 2.2 of \cite{goldstein2003equidistribution}]
For every $f\in C_c(\mathcal L_{n,1})$
\[\lim_{N\rightarrow +\infty} \int_{\mathcal L_{n,1}} f d\mathfrak h_{N} = \int_{\mathcal L_{n,1}} fd\mu_{n,1}\]
\end{theorem}

This result allows to approximate the Haar-Siegel probability measure by the measure obtained by fixing a big integer $N$, drawing uniformly a random element of the finite set $\mathfrak H_N$, then renormalizing by $N^{-\frac{1}{n}}$.

%%%%%%%%%%%%%%%%%%%%%%%%%%%%%%%%%
\section{Relative BDD}%%%%%%%%%%%
%%%%%%%%%%%%%%%%%%%%%%%%%%%%%%%%%

Let $G$ be a locally compact abelian group endowed with a metric $d : G\times G \rightarrow \mathbb R$, $H$ a finitely generated subgroup and $\mu$ a probability measure on $G/H$. Define the following problem
$$\mathcal P (G,H,\mu) : \text{Given } x \text{ s.t. }x+H \sim \mu , \text{ find argmin}_{a\in H} d(x,a).$$

In particular instances, this problem specializes to well studied problems. The generalization is motivated to highlight some subtle points : it can happen that one has to deal with codes while starting with a problem related to lattices. For example, this phenomenon arises in LWE because the quotient of the LWE lattice $\Lambda_q(A)$ by $q\mathbb Z^n$ is not a lattice but a finite group. 
\[\begin{array}{|c|c|c| c | c|} 
\hline
G             & H                          & d     &  \mu              & \mathcal P (G,H,\mu)\\
\hline
\mathbb R^n   & \Lambda                    & \ell^2 & \chi_e            & \text{CVP}     \\
\mathbb F_q^n & \mathcal C                 & d_H    & \text{uniform on fixed weight}    & \text{Decoding}\\
\mathbb Z^{\text{nrow}(A)}  & \Lambda_q(A) & \ell^2 &  A \chi_s +\chi_e & \text{LWE}     \\
\hline
\end{array}\]

Our work focuses on breaking $BDD(\Lambda)$ into a series of problems with the hope of decreasing the complexity :
\begin{itemize}
\item[$\bullet$] find $\Lambda_0$ such that $BDD(\Lambda_0)$ has quasi-polynomial complexity\footnote{Or just at least a better complexity than expected $BDD$ complexity.}.
\item[$\bullet$] apply the dual attack to reduce $BDD(\Lambda)$ to $BDD(\Lambda_0)$. 
%\item[$\bullet$] for $G = \Lambda/\Lambda_0$, $BDD(\Lambda)$, find $b\in B$ such that  is equivalent to 
\end{itemize} 

We also provide a rigorous statistical model for the attack. For the distinguishing step, we establish an exact deviation bound for the score which does not suppose independance on the terms of the sum. We explicitely link the maximization of the score function to the optimization of the p-value of the tests. The end of the work is devoted to a discussion of the parameters for which the attack is efficient, and comparison with previous work. 

Our result allows a modification of the attack as described in \cite{pouly2023provable}. As we don't need the independance of the summands in the distinguisher, we can use a sieve to generate our sample of short vectors $W$ in $\Lambda^\vee$. This is more efficient than using independant calls to a gaussian sampler iteratively. Moreover, it is not clear that the results in \cite{pouly2023provable} apply even if $W$ is sampled with $|W|$ calls to a sampler, since the different terms are not independant once the randomness of the target is taken into account\footnote{if $w,w'$ are independant and $t\sim D_\Lambda$, $\langle w , t \rangle $ and $\langle w' , t \rangle $ are not necesserily independant.}.   
 
The distinguishing step depends on a sublattice $\Lambda_0 < \Lambda$. One builds a finite set of short vectors $W\subset \Lambda_0^\vee$ and define the distinguisher
$$f_W(t) = \frac{1}{|W|} \sum_{w\in W} \cos (2\pi \langle w, t\rangle ).$$ 
This function is $\Lambda_0$-periodic and thus 
$$ t_0^* = \text{argmax}_{t\in V} f_W(t) = \text{argmax}_{t\in V/\Lambda_0} f_W(t+ \Lambda_0) $$
satisfies $t - t_0^*$ follows a BDD-distribution on $\Lambda_0$.

In the setting of LWE lattices, let us write $A  = [A_{dual} | A_{guess} ]$. With the notations as above, 
$$\Lambda_0 = \Lambda_{q}(A_{dual}) < \Lambda_q(A).$$ 
If the columns of $A$ are independant, we have that 
$$\Lambda_{q}(A) / \Lambda_{q}(A_{dual}) \cong \Lambda_{q}(A_{guess})$$ 
and when the attack is succesful, i.e. $s_{guess} = s_{guess}^*$, we get $t-As_{guess}^* = A_{dual}s_{dual} + e$.

In the setting where $\Lambda / \Lambda_0 = G$ is a finite abelian group (as in \cite{ducas2023does}), we get the problem $\mathcal P(G,\mu)$. In \cite{ducas2023does}, the optimisation of $f_W$ is done first with computing its Fourier transform, which for particular case of groups such as $(\mathbb Z / 2 \mathbb Z)^n$ can be done very efficiently. 

\subsection{Concentration inequalities for lipschitz functions of discrete gaussians}
In order to be able to apply the exact bounds for a LWE distinguisher, we prove concentration inequalities valid for discrete gaussians. 

\begin{lemma}
Let $X$ be a real random variable and $\varphi : \mathbb R \rightarrow \mathbb R$ be a differentiable function such that $t\mapsto \varphi'(t)\mathbb P(X>t)$ is integrable. The following formula holds
$$\mathbb E [\varphi(X)] = \varphi(a) + \int_{a}^{+\infty} \varphi'(t)\mathbb P(X > t) dt \quad \forall a \in \mathbb R.$$
If $\lim_{t\rightarrow -\infty}\varphi(t) =0$, the following also holds
$$\mathbb E [\varphi(X)] = \int_{-\infty}^{+\infty} \varphi'(t)\mathbb P(X > t) dt.$$
\end{lemma}

\begin{proof}
Applying the fundamental theorem of calculus,
$$\varphi(x)-\varphi(a) = \int_a^x\varphi'(t)dt$$
Evaluating the expectation and Fubini interversion theorem concludes the first part.
For the second part, we let $a$ tend to $-\infty$ to get :
$$\varphi(X) = \int_{-\infty}^X\varphi'(t)dt = \int_{-\infty}^{+\infty} \varphi'(t) 1_{\{X > t\}} dt.$$
and concludes in the same manner.
\end{proof}

We call a real random variable $X$ sub-gaussian if there exist constants $C , a>0$ such that 
$$\mathbb P( X > t ) \leq Ce^{-at^2} \quad \forall t >0.$$ 
If we want to keep track of the constants involved, we will call such a variable a $(C,a)$-sub-gaussian random variable.

\begin{theorem}
Let $(M,d)$ be a metric space, and $\xi$ a random variable with values in $M$, such that for every (or equivalently for one) point $m\in M$, the random variable $d(\xi, m)$ is $(C,a)$-sub-gaussian. Let $f : X\rightarrow \mathbb R$ be a lipschitz function with lipschitz constant bounded by $L>0$. Then  
$$\mathbb P(|f(\xi)  - \mathbb E[f(\xi)]| > t) \leq   \frac{C^2a^3}{\pi L^2} \cdot t^2 e^{-\frac{ L^2}{8a}t^2} .$$
\end{theorem}

\begin{proof}
Let $\tilde\xi$ an independant copy of $\xi$, $m\in M$ and $\lambda>0$. We have 
\[\begin{split}
\mathbb P(|f(\xi)  - \mathbb E[f(\xi)]| > t) & \leq e^{-\lambda t} \mathbb E[e^{\lambda |f(\xi)  - \mathbb E[f(\xi)]| } ] \\ 
		& \leq  e^{-\lambda t} \mathbb E[\exp({\lambda \mathbb E[|f(\xi)-f(\tilde \xi)| \ | \ \xi] })] \\
		& \leq  e^{-\lambda t} \mathbb E[e^{\lambda |f(\xi)-f(\tilde \xi)| }] \\
		& \leq  e^{-\lambda t} \mathbb E[e^{\lambda Ld(\xi , \tilde \xi) }] \\
		& \leq e^{-\lambda t} \mathbb E[e^{\lambda Ld(\xi , m) }]^2 \\
\end{split}
\]
The preceeding lemma applied to $\varphi(u) = e^{L\lambda u}$ and $X=d(\xi,m)$ ensures that
\[\begin{split}
\mathbb E [e^{\lambda L d(\xi , m)} ] & = \lambda L \int_{-\infty}^{+\infty} e^{\lambda L u} \mathbb P(d(\xi,m) > u) du \\
	& \leq C\lambda L \int_{-\infty}^{+\infty} e^{\lambda L u-au^2} du \\
	& \leq C\lambda L e^{a\frac{\lambda^2L^2}{4a^2}}\int_{-\infty}^{+\infty} e^{-a(u-\frac{\lambda L}{2a} )^2} du \\
	& \leq C\lambda L \sqrt{\frac{a}{\pi}} e^{\frac{\lambda^2L^2}{4a}} .\\
\end{split}\]
Replacing in the last inequality, we get
\[\begin{split}
\mathbb P(|f(\xi)  - \mathbb E[f(\xi)]| > t) & \leq C^2 L^2 \frac{a}{\pi} \cdot \lambda^2 e^{\frac{ L^2}{2a}\lambda^2 -\lambda t} \\ 
\end{split}
\]
Setting $\lambda = \frac{at}{L^2} $, we get the result. I did not dare minimizing in $\lambda$. 
\end{proof}

Suppose now that there exist constants $C , a , t_0>0$ such that 
$$\mathbb P( X > t ) \leq Ce^{-at^2} \quad \forall t >t_0.$$
The same method as in the previous result yields 
$$\mathbb E [e^{\lambda L d(\xi , m)} ] \leq e^{\lambda L t_0}  + C\lambda L \sqrt{\frac{a}{\pi}} e^{\frac{\lambda^2L^2}{4a}} $$
hence 
$$\mathbb P(|f(\xi)  - \mathbb E[f(\xi)]| > t) \leq e^{-\lambda t} \left(e^{\lambda L t_0}  + C\lambda L \sqrt{\frac{a}{\pi}} e^{\frac{\lambda^2L^2}{4a}}\right)^2 $$
If $\lambda = \frac{at}{L^2} $, we get
\[\begin{split}
\mathbb P(|f(\xi)  - \mathbb E[f(\xi)]| > t) & \leq e^{-\lambda t} \left(e^{\lambda L t_0}  + C\lambda L \sqrt{\frac{a}{\pi}} e^{\frac{\lambda^2L^2}{4a}}\right)^2 \\
					& \leq \left( e^{at_0^2}e^{-a(\frac{t}{L} - t_0)^2} + \frac{C^2a^3}{\pi L^2} \cdot t^2 e^{-\frac{ L^2}{8a}t^2} \right)^2 \\
\end{split}
\]

Let us set our notation for the gaussian function of parameter $s>0$, 
$$\rho_s(v)=\exp(-\pi\frac{\|v\|^2 }{s^2} )\quad \forall v\in \mathbb R^n.$$
The continuous gaussian measure\footnote{We denoted the Lebsgue measure of dimension $d$ by $m_d$.} is 
$$\gamma_s(A) =  \int_A s^{-n}\rho_s(x)dm_d(x) ,$$
whereas the discrete gaussian on $\Lambda$ is the probability measure 
$$D_{\Lambda , s} = \rho_s(\Lambda )^{-1} \sum_{v\in \Lambda} \rho_s(v)\delta_v. $$
Recall the following corollary of a theorem of Banaszcyk theorem (lemma 1.5 in \cite{banaszczyk1993new}).

\begin{theorem}
Let $\Lambda$ an euclidean lattice of rank $d$ and $\xi \sim \mathcal D_{\Lambda , s}$,
$$\mathbb P(\|\xi - v\|> r ) \leq \exp(-\pi (\frac{r}{s} -\sqrt{\frac{d}{2\pi}})^2) \quad \forall r >\sqrt{\frac{n}{2\pi}}s $$
\end{theorem}

Let $W\subset \Lambda^\vee$ be a finite subset, and define 
$$f_W(t) = \frac{1}{|W|}\sum_{w\in W} \cos(2\pi \langle w , t \rangle) \quad \forall t \in \ell^2_n.$$
Let us denote by $L$ the positive number $2\pi\max_{w\in W} \|w\|_2$. Then
\[\begin{split}
|f_W(x) -f_W(y)| & \leq \frac{1}{|W|}\sum_{w\in W} |\cos(2\pi \langle w , x \rangle ) -\cos(2\pi \langle w , y \rangle) | \\
		 & \leq \frac{2\pi}{|W|}\sum_{w\in W} \|w\|_2 \|x-y\|_2\\
		 & \leq L\|x-y\|_2\\ 
\end{split}\]
i.e. $f_W$ is a $L$-Lipschitz function.  

If one uses a sieve to sample vectors from $W$, we can give the following bounds 
$$2\pi gh(\Lambda) \leq L \leq 2\pi\sqrt{\frac{4}{3}}gh(\Lambda) $$

%%%%%%%%%%%%%%%%%%%%%%%%%
%%%%%%%%%%%%%%%%%%%%%%%%%
\bibliographystyle{plain}
\bibliography{biblio}
\end{document}































