\documentclass{article}

\usepackage{amsfonts}
\usepackage{amsmath}
\usepackage{amsthm}

\setlength\parindent{0pt} %no indentation for paragraph

\newtheorem{definition}{Definition}
\newtheorem{theorem}{Theorem}
\newtheorem{proposition}{Proposition}
\newtheorem{lemma}{Lemma}

\title{Lattice based cryptography}
\author{Clément Dell'Aiera}

\begin{document}
%%%%%%%%%%%%%%%%%
%%%%%%%%%%%%%%%%%
\maketitle

Denote by $\mathfrak R$ a ring of the type $\mathbb Z[X]/(\varphi)$ with $\varphi$ a monic polynomial, and $\mathfrak R_q$ its quotient by the ideal $q\mathfrak R$.
 
\begin{center}\fbox{\begin{minipage}{9cm}
\textbf{M-SIS :} parameters $(q,m,n,\beta)\in \mathbb N_{>0}^3\times \mathbb R  $ such that $m>n$.\\
Given $A\leftarrow \mathfrak R_q^{n\times m}$, find $z\in \mathfrak{R}^m$ such that $$0< \|z\| \leq \beta$$
$$Az^T = 0.$$
\end{minipage}} \end{center}

\begin{center}\fbox{\begin{minipage}{9cm}
\textbf{M-LWE :} parameters $(q,m,n,\nu)\in \mathbb N_{>0}^3\times P( \mathfrak R)  $ such that $m>n$.\\

(Decisional) Given $(A,As+e)\in \mathfrak{R}_q^{n\times m}\times \mathfrak{R}_q^m$, distinguish between $(A,b)\leftarrow \mathfrak{R}_q^{n\times m}\times \mathfrak{R}_q^m$ and $(A,As+e)$ where $s\in \mathfrak R_q^n$ is a secret and $e\leftarrow \nu^{\otimes m}$.\\

(Computational) Given $(A,As+e)\in \mathfrak {R}_q^{n\times m}\times \mathfrak {R}_q^m$, where $s\in \mathfrak R_q^n$ is a secret and $e\leftarrow \nu^{\otimes m}$, find $s$.
\end{minipage}} \end{center}

\begin{center}
\fbox{\begin{minipage}{9cm}
\textbf{BDD($\alpha$) :}\\
Given a lattice $\Lambda<V$, a target $t\in V$ such that $d(t,\Lambda)\leq \alpha \lambda_1 (\Lambda)$, find $v\in \Lambda\cap B(t,\alpha\lambda_1)$.
\end{minipage}} 
\end{center}

%\documentclass{article}

\usepackage{amsfonts}
\usepackage{amsmath}
\usepackage{amsthm}

\usepackage{hyperref}

\setlength\parindent{0pt} %no indentation for paragraph

\newtheorem{definition}{Definition}
\newtheorem{theorem}{Theorem}

\title{Notes sur les attaques duales}
\author{Clément Dell'Aiera}

\begin{document}
%%%%%%%%%%%%%%%%%
%%%%%%%%%%%%%%%%%
\maketitle

On étudie les Dual-Sieve-FFT attacks sur le problème BDD.

Pour Clémence,

(0) Notions sur les réseaux : définitions, exemples de familles de réseaux utilisés en cryptographie, protocoles existants (en particulier ceux soumis au NIST), lien avec la cryptographie post-quantiques (et enjeux d'icelle), problèmes calculatoires liés aux réseaux (BDD,  LWE, NP, réduction du cas moyen au pire cas etc), attaques duales, algorithmes de réduction (Gauss en dimension 2, et LLL puis BKZ), cribles et énumération. Peut-être : notion sur les tests statistiques, ainsi que sur la complexité algorithmique.  

(1) Donner les détails du lien entre BDD et LWE : quel instance particulière de BDD et avec quel réseau donne un problème LWE.

(2) Implémentations : fpLLL et G6K en Sage, implémenter un distingueur et l'attaque.

(3) Un exemple peut-être intéressant : on se donne un réseau $\Lambda_0$ sur lequel BDD est facile à résoudre, et on génère un sur-réseau $\Lambda$ d'indice fini aléatoirement\footnote{Par exemple, on génère aléatoirement une suite d'entiers $d_k$ qui se divisent successivement et on prend pour base de $\Lambda$ une base de $\Lambda_0$ multipliée par $diag(d_k)$ cf section 3.2 de \cite{DucasPulles}.}. On applique alors l'attaque généralisée de Ducas-Pulles. Par exemple si le quotient $G$ est un code aléatoire difficile, est ce que BDD sur $\Lambda$ est aussi fort que le décodage par syndrome sur $G$?

\section{Revue de la littérature}

\subsection{Laarhoven Walter (2021)}

Modélisation statistique du Distingueur et dérivation de l'estimateur de Neyman-Pearson. Comparaison avec l'estimateur de Aharonov-Regev et l'estimateur le plus simple (la somme des cosinus). 

La notion de distingueur optimal est à éclaircir. Les auteurs semblent se reposer sur le nom de l'estimateur pour en conclure qu'il est "le meilleur", sans pour autant détailler quels critères l'estimateur de Neyman-Person optimise (l'erreur de seconde espèce parmi les estimateurs de risque fixé, je crois).

Une question intéressante : calculer l'erreur de seconde espèce pour le test associé (on peut comparer AR et NP). Plus tard dans ces notes, on explique que maximiser le score sur un échantillon revient à effectuer le test de distinction 'H0 : le target suit une loi uniforme' vs 'H1 : le target suit une loi LWE' pour chacun des éléments de l'échantillon pour ensuite séelctionner l'élement dont le test a la meilleure p-value.

\subsection{Guo Johansson (2021)}

Computation of the statistical distance between the probability distributions of $\langle w , t \rangle \pmod{1}$ for $t$ uniform and gaussian.  

\subsection{Matzov (2022)}

%%%%%%%%%%%%%%%%%%%%%%%%%%%%%%%%%%
\subsection{Ducas Pulles (2023)}%%
%%%%%%%%%%%%%%%%%%%%%%%%%%%%%%%%%%

\subsubsection{} Page 3 : DP remarque que l'évaluation de l'attaque repose sur l'hypothèse d'indépendance des termes de la somme du score (Independance Heuristic). Ils pointent aussi du doigt que si le bruit d'une erreur LWE est trop grande (dépasse la Gaussian Heuristic, i.e. si la distance au réseau est plus grande que le rayon de recouvrement du réseau) alors il ne devrait pas être possible de distinguer LWE d'un tirage uniforme.\footnote{J'ai l'impression que leur argument (tout à fait valable) peut se retourner contre leurs analyses. Les scores sont en effet non indépendants, pourtant ils le supposent plus tard lors de leur modélisation, lorsqu'ils affirment qu'un tirage BDD parmis un grand nombre de tirages uniforme n'aura pas forcéement le score le plus elevé. Cela est vrai, en supposant independance dans l'échantillon.}

Le dernier argument est intéressant : ils affirment que distinguer une instance BDD d'un GRAND NOMBRE de cibles uniformes (indépendantes) est voué à l'échec car l'échantillon uniforme va avoir tendance à recouvrir le domaine fondamental du réseau et va donc produire des points arbitrairement proches d'icelui. Cet argument repose toutefois fortement sur l'indépendance des termes du score, ce qui est étonnant : les auteurs pointent du doigt ce même défaut dans les travaux antérieurs.   

Si l'on veut distinguer un tirage BDD d'un tirage uniforme, un problème que l'on va rencontrer est le suivant. La loi uniforme va recouvrir l'intervalle, et si l'on ne tire qu'un seul tirage BDD contre un grand nombre de tirage INDEPENDANTS uniformes, alors l'un de ces tirages finira fatalement par être plus proche du réseau que le tirage BDD.

Pourtant l'attaque fonctionne en pratique. Les auteurs s'en sortent en affirmant que leurs critiques ne s'appliquent qu'à un régime (entendre un choix de paramètres) théorique qui n'est pas celui choisi en pratique par les travaux plus anciens.

\subsubsection{}

Page 6 est 'prouvée' l'identification entre dual au sens des réseaux et dual unitaire au sens de théorie des représentations des groupes localement compacts. On rappelle ce résultat standard : l'application 
$$w \in \Lambda^\vee \mapsto \left(t\in V \mapsto \exp(2i\pi \langle w,t \rangle)\right)$$ 
induit un isomorphisme $$\Lambda^\vee \cong \widehat{V/\Lambda}.$$ 

Les auteurs rappellent page 7 l'heuristique gaussienne : pour un réseau aléatoire de covolume 1, $\lambda_1$ peut être approximé par $\mathfrak{gh}_n = vol(B_n)^{\frac{1}{n}}$. De manière générale,
$$\lambda_k \approx \mathfrak{gh}_n \cdot k^{\frac{1}{n}}.$$ %\simeq 

Cette heuristique se trouve aussi sous la forme suivante : si $\Lambda$ est un réseau aléatoire, et $\Omega$ une partie mesurable de $V$, alors 
\[\frac{Vol(S)}{Covol(\Lambda)} \approx |S\cap \Lambda|.\]
J'imagine que par réseau aléatoire est sous-entendu l'unique mesure de probabilité invariante\footnote{Tout quotient d'un groupe localement compact par un réseau possède une unique mesure de probabilité invariante} sur $GL(n,\mathbb R) / GL(n,\mathbb Z)$. Dans \cite{Shen}, on étudie la mesure de probabilité donnée par 
\[\mathbb P(\Lambda = \Lambda_q (A) ) = \frac{1}{|M_n(\mathbb Z / q\mathbb Z)|}.\]
On peut se demander si cette probabilité est naturelle. On voit que $\Lambda_q(A) = \Lambda_q(A')$ ssi 
\[\begin{pmatrix} I_n & 0 \\ A & qI_m \end{pmatrix} GL(m+n,\mathbb Z) = \begin{pmatrix} I_n & 0 \\ A' & qI_m \end{pmatrix}GL(m+n,\mathbb Z)\]
i.e. il existe $g\in M_{m,n}(\mathbb Z), h \in M_{n,m}(\mathbb Z)$ telle que 
\[\begin{split} 
A-A' & = qg \\
gh   & = qI_m \\
\end{split}\]	 

 
\subsubsection{}

Je ne comprends pas l'intérêt de
$$\mathbb P( \forall i , f_W(t_{bdd}) > f_W(t^{(i)}_{unif}) )   $$
dans l'étude de la puissance du test. Note : Je pense avoir compris. C'est la probabilité de distinguer une target LWE dans un échantillon contenant des uniformes sur le domaine fondamental. On ne parle pas de l'existence d'un distingueur, mais de la probabilité de réussite du test consistant à choisir le tirage dont le score est le plus élevé.

\subsubsection{}

La partie FFT de l'attaque est formalisée de la manière suivante. On se donne un sous-réseau $\Lambda_0 < \Lambda $ d'indice fini, et on note $G$ le groupe quotient (fini). Les auteurs affirement que le problème $BBD_{\Lambda}(t)$ se réduit aux problèmes $BDD_{\Lambda_0}(t+g) \forall g \in G$. Plutôt que d'évaluer chaque score $f_{W_0}(t+g)$ séparément, on évalue la transformée de Fourier de la fonction $g \mapsto  f_{W_0}(t+g)$, qui se calcule de manière analytique, puis on lui applique la transformée de Fourier inverse.

Il faudrait détailler ce passage. J'en comprends que l'on suppose savoir 

\subsubsection{}
Je ne comprends pas la remarque "Randomized Sparsification" à la fin de la page 14.	

\subsubsection{}
Trouver une référence pour le théorème page 18, qui donne une version alternative (et réelle) de l'heuristique gaussienne, affirmant que si $t\sim \mathcal U(\mathbb R^n/\Lambda)$ et si $V\subset \mathbb R^n$ est mesurable, 
$$\mathbb E|(V -t)\cap \Lambda | = \frac{vol(V)}{det(\Lambda)}$$
On peut le montrer de la manière suivante. On définit la forme linéaire continue sur l'espace des fonctions continues à support compact sur $\mathbb R^n$ par 
$$f\mapsto \sum_{v\in \Lambda} \mathbb E[f(v+t )]. $$
La mesure qui représente cette forme linéaire est invariante par translation, elle est donc proportionnelle à la mesure de Lebesgue. (En fait on raisonne par pull back de la mesure de Lebesgue par l'isomorphisme $\mathbb R^n \cong \Lambda \times \mathbb R^n /\Lambda$)\qed 

\subsubsection{}

La discussion page 18 (The Contradiction) a l'air raisonnable\footnote{On peut reprocher aux auteurs d'être trop vague : le résultat montre que le distingueur basé sur la maximisation du score sur un échantillon contenant un nombre très grand de targets uniformes indépendants et un target bdd ne fonctionne pas dans tous les régimes, notament si la variance du bdd est trop grande et/ou si le nombre de targets uniformes est trop grand. Le lien avec l'attaque duale n'est pas clair : l'hypothèse d'indépendance ne tient pas.}. Même si l'on souhaite utiliser des grande déviations, les bornes obtenues vont dépendre de la constante de Lipschitz du distingueur. Dans le cas du distingueur AR, celle-ci est dominée par la taille maximale de $W$, et le résultat de l'argumentation de Ducas-Pulles ne contradit pas le nôtre : on ne peut pas distinguer n'importe quoi. Le problème est de distinguer une target LWE située à une distance $r\cdot GH(n)$ d'une uniforme sur un domaine fondamental?? 

Voici une formalisation de cette argument. Soit $t_1,\cdots ,t_n$ des variables aléatoires indépendantes satisfaisant à :
\begin{itemize}
\item[$\bullet$] il existe $i_0$ telle que $t_{i_0}$ suive une loi BDD.
\item[$\bullet$] pour tout $i\neq i_0$, $t_i$ suivent une loi uniforme.
\end{itemize}
On note $i^* = \text{argmin}_i f(t_i) . $
Alors $\lim_{n\rightarrow \infty } \mathbb P(i_0\neq i^* ) =1$. Cet argument est valide, cependant il ne s'applique pas tel quel à l'attaque : les targets ne sont pas indépendants : $t_s = t - As $.

La première partie de 4.2 pousse les arguments de Matzov et GuoJohannson jusqu'à une supposée contradiction : si l'attaque fonctionne, on arrive a distinguer un target BDD parmi un nombre exponentiel de BDD uniformes (indépendants? l'hypothèse semble nécéssaire au raisonnement mais n'est précisé nul part).  

Le lemme 8 montre que si $det(\Lambda) = 1$ et $t\sim\mathcal U(\mathbb R^n /\Lambda)$ alors
$$\mathbb P( d(t , \Lambda ) \leq r\cdot GH(n) ) =  r^n \quad \forall r \text{ t.q. } r \cdot GH(n) < \frac{\lambda_1^{(\Lambda)}}{2}$$
Si l'on a une seule target BDD $t_{bdd}$ et $T>>1$ target uniformes indépendantes $t^{(i)}_{unif}$, alors la probabilité qu'aucune des target uniformes ne tombent dans une boule de rayon $r\cdot GH(n)$ centrée en le point du réseau est $(1-r^n)^T \sim_{T\rightarrow \infty} e^{-\alpha^n}$. L'espérance de la longueur du target BDD est $\sigma \sqrt n $ : avec une variance assez grande, le distingueur ne peut pas espérer utiliser la distance pour conclure.

\subsubsection{}

Une question : est-il possible que bien que des targets uniformes soient plus proches du réseaux que le target BDD, le distingueur AR discrimine correctement? C'est abordé dans Discusion page 20, et les auteurs répondent par la négative.

En pratique, les targets uniformes ne sont pas indépendants : l'ensemble des targets (BDD y compris) est constitué des vecteurs 
$$ t = A_0 s_0 +A_1 \tilde s_1 + e \quad (LWE) $$

$$ t = B (x_0 + \tilde x_1 ) + e \quad (BDD) $$

Donc les $t^{(i)}_{unif} $ sont exactement les $t-B\tilde x_1$ où les $\tilde x_1$ on été mal devinés. Ils ne sont pas indépendants (ils dépendent du même aléas $x_0$ et $e$)  

Waterfall phenomenon : à investiguer. 

Une question : lors de l'attaque duale, les tirages uniformes concernent les mauvais guess's. Ils ne sont pas indépendants: on observe
$$ U_k = A_0 s_0 + A_1 (s_1 - s_guess_k) + e $$
contre 
$$ BDD = A_0 s_0 + e $$

Calcul explicite de l'espérance et de la variance de $ cos( 2\pi \langle w , t\rangle )$ lorsque $t$ suit une loi uniforme $\mathcal U(\mathbb T_\Lambda)$ (on obtient $(0,\frac{1}{2})$) et si $t\sim \mathcal N_{\ell^2_n}(0 , \sigma^2 I_n) \pmod{\Lambda}$ (on obtient $(e^{-2\pi² \sigma^2 \|w\|^2} , \frac{1}{2} - \Theta(e^{-4\pi^2\sigma^2 \|w\|^2}))$.

\subsection{Pouly Shen (2024)}

Le papier étudie le distingueur dans le cas des réseaux associés au problème LWE, i.e. pour $A\in (\mathbb Z /q\mathbb Z )^{m\times n}$ avec $m\geq n$ et rang$(A)=n$,

$$\Lambda_q(A) = \{x\in \mathbb Z^m | \exists y \in Z^n \text{ s.t. }Ay = x \pmod q \} ,$$
$$\Lambda_q(A)^\perp = \{x\in \mathbb Z^m | Ax = 0 \pmod q \} = q \Lambda_q(A)^\vee.$$
Les deux réseaux sont $q$-congruents. On suppose toujours $m\geq n$ et $A$ de rang maximal (on peut toujours se ramener à ce cas). Seul le cas $m>n$ est intéressant, car si $A$ est inversible, $\Lambda_q(A) = \mathbb Z^m$. En écrivant $A$ par blocs 
$$A =\begin{pmatrix} A_0 \\ A_1\end{pmatrix},$$
et si on suppose $A_0$ inversible, 
$$B=\begin{pmatrix} I_n & 0 \\ A_1A_0^{-1} & qI_{m-n}\end{pmatrix} $$
est une base de $\Lambda_q(A)$ et $(B^{-1})^t$ une base du dual.

Mentionne dans l'introduction un papier de Meyer-Hifliger \& Tillich, Rigorous foundations for dual atacks in coding theory, qui prétend démontrer un lien entre attaques duales sur les réseaux et les codes. L'article prétend que le lien n'est pas si évident. Il serait intéressant de creuser.

Page 3 : "contribution of this paper is to mak completely clear (Theorem 6) under what choice of parameters the attack works, without any statistical assumption." Cela est faux : ils ont bien défini un modèle statistiques qui a des hypothèses (par définition) d'autant plus qu'ils supposent l'indépendance des targets.

En corollaire : résultats sur les réseaux congruents (j'appelle congruents les "q-ary lattices", par analogie avec les sous-groupes de congruences).

Qu'est ce que le Monte Carlo Markov Chain discrete Gaussian sampler? Smmoting parameters : sampler pour une 'variance' au dessus est plus facile, sampler en dessous est plus difficile (assez bas et cela revient à résoudre SVP qui a été démontré dans NP par Atjai 1998)

Utilisation d'intervalles de confiances exacts via l'inégalité de Hoeffding pour le distingueur au lieu d'un intervalle asymptotique donné par le TCL (et une approximation gaussienne)

Page 8, je ne comprends pas 'In fact, a proper analysis shows that the minimum value of N needed for convergence is roughly the same as the value needed for distinguishing, so this is no small detail.'

Calcul de \cite{19} : $\Phi_X(\frac{k}{q}) = \mathbb E[e^{2i\pi k X/q}] $ vaut
\begin{itemize}
\item si $X\sim \mathcal U(\mathbb Z/q)$, $\Phi_X(\frac{k}{q})  = 0$ pour tout $k$,
\item si $X\sim \mathcal D_{\mathbb Z/q , s}$, $ e^{-\pi s^2k^2 / q^2} \leq \Phi_X(\frac{k}{q})  \leq 2e^{-\pi s^2k^2 / q^2}$ pour $k\in [-\lfloor q/2\rfloor , \lfloor q/2\rfloor ]$.
\end{itemize}

2.5 : le papier s'intéresse aux réseaux congruents. En temps qu'ensemble statistique, il est muni de la probabilité induite par le tirage uniforme d'une matrice de rang plein $A\in \mathbb Z_q^{n\times k}$ par rejet (donc uniforme sur les matrices de rang maximal). Tout réseaux congruents est isomorphe à un $\Lambda_q(A)$. Quelle mesure obtient-on? Les résultats de l'articles s'étendent-ils à tous les réseaux via la mesure de probabilité invariante de l'espace des réseaux?  

L'heuristique gaussienne affirme que $\lambda_1(\Lambda)$ est environ $GH(n) = \left( \frac{vol(B_n)}{det(\Lambda)}\right)^{-\frac{1}{n}}$. Elle peut être précisée en un théorème sur la distribution de $\Lambda_1(\Lambda)$ en fonction d'une mesure de probabilité sur $\Lambda$. Pouly-Shen prouvent (Corolary 2 page 10) que si $A\sim \mathcal U(\mathbb Z^{k \times n}_{rk = \max (k,n)})$ et $\Lambda = \Lambda_q(A)$, alors
$$ \mathbb P\left( \lambda_1( \Lambda ) \leq \varepsilon \mathfrak{gh}(\Lambda) \right) \leq q\varepsilon^n  $$

Description de l'attaque duale page 13.

Je ne comprends pas "The limitation ... is the requirement to compute one short vector for each tuple of m LWE-samples ... this is necessary to ensure statistical independance of the variables...". Ici on attaque LWE, et on peut se permettre de faire varier le réseau (en faisant varier la taille de la matrice $A\in \mathbb Z^{Nm \times n}$. Les vecteurs de $W$ sont samplés par blocks : on obtient des $w$ orthogonaux (le $k$-ième est supporté sur les coordonnées de $km$  à $(k+1) m -1$. Il reste toujours un problème : les variables aléatoires $\langle w , t \rangle $ ne sont pas indépendantes, elles ne le sont que sachant $t$ (ou alors si les composantes de $t$ sont indépendantes, ce qui n'a pas l'air d'être vrai pour les gaussiennes discrètes, sinon il serait simple de les simuler en se ramenant à la dimension 1).  

page 14 -15 : justification de l'utilité du distingueur $g_W$. En notant $f_{L,s}(x) = \frac{\rho_s(x+L)}{\rho_s(L)}$ et 
$$D_{L , s}  =\frac{1}{\rho(L)}\sum_{v\in L} \rho_s (v)\delta_v$$ 
on a 
\[\begin{split}
s_{guess} & = \text{argmin}_s d(t - A_{guess} s , L ) \\
	& \approx \text{argmin}_s \  f_{L, 1/s}(t - A_{guess} s) \\
	& \approx \text{argmin}_s \ g_W(t - A_{guess} s ) \\
\end{split}\]

Dans 4.4 Complexity estimates : description de la complexité de la génération de vecteurs courts du réseau dual. Important : l'indépendance des vecteurs courts du réseau dual est garantie par la méthode de sampling utilisée (MCMC), c'est ici une modification par rapport aux attaques duales de Matzov ou GuoJohannson par exemple.

\subsubsection{Discussion avec Yixin 12/07/2024}

La zone d'application de l'attaque correspond à \[\frac{1}{2}\lambda_1(\Lambda_q(A)) > \|e\|\]
ce qui garantie l'unicité d'une solution de LWE. La négation de cette condition redonne la zone de 'contradiction' de l'article de Ducas-Pulles. Pour Yixin, le sieving appliqué pour sampler $W$ restreint l'attaque à un sous-réseau $\Lambda_0$ de $\Lambda$. Il est donc possible que bien que l'on ne soit pas dans la condition $\frac{1}{2}\lambda_1(\Lambda) > \|e\|$, on ait
$\frac{1}{2}\lambda_1(\Lambda_0) > \|e\|$. Remarque : c'est vrai que si $W$ est par exemple contenu dans le réseau $\Lambda_0$ engendré par $(q_1 b_1,\cdots , q_n b_n )$ avec $(b_1,\cdots , b_n )$ une base de $\Lambda$, on obtient $\lambda_1(\Lambda_0) \geq \min q_i \lambda_1(\Lambda)$.

Yixin ne voit pas les termes du score comme des variables aléatoires. L'idée est que la fonction $f_W$ est une approximation d'une gaussienne de la distance au réseau. La maximiser revient à minimiser cette distance et donc à résoudre le problème LWE associé.
 
Les articles de ling et Wang (\cite{wang2017geometric} \cite{wang2019lattice}) détaille une méthode de sampling de gaussienne discrète sur les réseaux utilisant les méthodes de Monte-Carlo. Là où sampler par GPV donne un algorithme polynômial qui ne permet pas de descendre en dessous du smoothing parameter du réseau, la méthode basée sur le MCMC permet de le faire en étant une tout petit peu plus que polynômial.
 
\section{Sieving}
 
06/09/2024 : Depuis la rentrée, installation de g6k (et fpylll) et montée en compétence sur le sieving. 
Ref : \\
thèse de Postlewaite, \\
Ducas A few dimensions for free\\
Papier g6k, 
plus le talk \url{https://www.youtube.com/watch?v=g4fGalYrvAI&t=255s} de Elena Kirshanova.

\section{Idées}

(1) Test d'indépendance du chi-deux sur des samples de $\cos(2  \pi * \langle w , t \rangle)$ lorsque t suit une distribution $\mathcal U(\mathbb Z_q)$ ou une $A * s + e$ avec $s$ et $e$ des gaussiennes 
 
(2) Marche aléatoire pour sampling sur $W$ et théorème limite associé pour intervalle de confiance (Azuma = Hoeffding pour les martingales) 
 
(3) Autre fonction de score: la fonction somme des $\cos(2 \pi \langle w,t\rangle )$ est une approximation de la transformée de Fourier d'une mesure de probabilité symétrique sur le dual

(4) Utiliser G6K pour le sampling

\section{Remarques}   

(1) Ducas a fait des experimentations numériques en faisant du sieving avec G6K. Il affirme que les targets As + e, qui devraient etre des gaussiennes, n'en ont pas l'air (a voir probleme dejà connu dans les codes cf 6.1 Waterfall phenomenon

(2) Regarder ce que donnent des bases BKZ du dual qui seraient moins bonnes (generer les vecteurs de depart "un peu comme on veut")

(3) Exhiber un cas particulier qui ne se passe pas correctement (comme dans Ducas-Pulles) pour le corriger

%%%%%%%%%%%%%%%%%%%%%%%%%
%%%%%%%%%%%%%%%%%%%%%%%%%
\bibliographystyle{plain}
\bibliography{biblio}
\end{document}


\section{To do}

How to sample dual vectors? It seems the fashionable way consists of applying BKZ reduction to the lattice basis, and keeping the first $k$ vectors, with $k$ a suitable parameter. The sample consists then of all vectors 
$$\sum x_i b_i^{\vee}$$
with $-a \leq x_i \leq a$ (with $a$ again a suitable parameter).  

What if we randomly sample from short vectors in $\Lambda^\vee$? Of simulate a random walk 
$$\xi_{k+1} = \xi_k +\varepsilon_k$$
where $\varepsilon_k$ follows a uniform distribution on the set of first BKZ-reduced vectors of a dual basis? Maybe the test statistics is more prone to analysis via ergodic theorems? (Instead of using a central limit theorem on sample lacking independance).

\section{Results on lattices}

Concentration inequalities for lipschitz functions on discrete gaussian on lattices.

Distribution of $\lambda_1(\Lambda)$ for $\Lambda \sim \mu$, where $\mu$ is the unique inariant provavility measure on $GL(\mathbb Z ) \backslash GL(n,\mathbb R) / O(n) $.

\subsection{Independance}

For a subset $W\subset \Lambda^\vee$, let us set $X_W = (e^{2 i \pi \langle w , t \rangle})_{w\in W}$. All the previous work on the dual attacks seems to use the statistic
$$\frac{1}{|W|} \sum_{w\in W } \cos( 2\pi \langle w , t\rangle ) $$
and derive statistical tests based on either asymptotic confidence intervals relying on the Central Limit Theorem (\cite{ducas2023does}) or the Hoeffding inequality (\cite{pouly2023provable}). All these methods rely on the assumption that coordinates of $X_W$ are independent, which is not the case\footnote{More formally, they all belong to the algebra of measurable functions w.r.t. the $\sigma$-algebra generated by $t$.}: they are all function of the same variable $t$.

Yet, if one draws different samples of $X_W$, one will directly detect a different behaviour depending on the law of $t$. It might be that the coordinates are not far from being independant (if the set $W$ is chosen well). 

In order to understand the phenomenon at hand, here is a toy model : let $U_0, U_1$ be random variables following a uniform law on $(0,1)$ and a gaussian law $\mathcal N(0,s)$ respectively, and $b$ be a random Bernoulli varaible of parameter $p$. Set 
$$X_n = e^{2i\pi n U_b}\quad \forall n \in \mathbb N_{>0}. $$
Then the variables are dependent, yet uncorrelated conditionnaly to the event $\{b = 0\}$.      

For two real valued random variables $X$ and $Y$, the statistic 
$$ n^{\frac{1}{2}}\sup_{x,y}|\hat F_n^{(X,Y)}(x,y) - \hat F_n^{(X)}(x) \hat F_n^{(Y)}(y)|$$
provides a test of independance. We thus could offer (one has to do the numerical experiments) a reasonable explanation as to why the coordinates of $X_W$ seems independant by computing the $p$-value of the test (for the toy model as well).

While not independent, the coordinates of the random vector $X_W$ have a distribution depending on that of $t$. In particular, one can hope to distinguish their laws under the hypothesis where $t$ is drawn according to a LWE-distribution against the alternative where it is drawn as a uniform random variable on a fundamental domain of the lattice. We design a non-parametric test 
$$ H_0 : \nu_W = \mu \quad \text{vs} \quad H_1 : \nu_W \neq \mu $$ 
where $\nu_W $ is the distribution of $X_W$ and $\mu$ the uniform probability on the circle. 

Let $F_0$ be the cumulative distribution function of the random variable $\cos(2\pi U)$ where $U\sim \mathcal U(0,1)$, $\hat F_W$ the empirical cumulative distribution of the coordinates of $X_W$, and
$$T_W = |W|^{\frac{1}{2}}\sup |\hat F_W(x) - F_0(x)|.$$
If $|W|$ is big enough, the statistic ... no this assumes independance of the $\langle w , t \rangle $ again!

As for implementation, the $\sup$-norm is quite impractical to compute, thus one might rely on the Cramér-Von Mises statistic
$$T^2 = \int |\hat F_n(x) -F(x)|^2 dF(x),$$
Another idea would be to try and find the maximum of 
$$T_x = |\hat F_n(x) -F(x)|^2 $$
by gradient descent. As numerical experiments show, $\cos(2 \pi\langle w , t\rangle)$ have a distribution very close to $\cos(2\pi U)$ when $t$ follows a uniform law, whereas it follows a distribution concentrated around $1$ in the LWE-case. One might thus follow the following algorithm.

\begin{center}\fbox{\begin{minipage}{7cm}
Set $x = \frac{1}{2}$, $\varepsilon$ such that $T > \varepsilon$ is a rejection domain of desired risk, $N_{max}$ the limit number of loops.\\
Compute $T_x$. \\
Loop for $N_{max}$ times:\\
If $T_x > \varepsilon$, return $H_1$, break\\
Else $x \leftarrow x +\eta\nabla T_x$\\
End loop\\
Return $H_0$.
\end{minipage}}
\end{center}

As a final remark : it could be a good idea to replace the circle $\mathbb S^1 = \{z\in \mathbb C : |z|=1\}$ and the exponential, cosine and sine functions by $\mathbb R / \mathbb Z$ (identified as $(-\frac{1}{2} , \frac{1}{2})$) and the floor function, as to avoid numerical errors coming from the use of transendental functions and numbers such as $\pi$. That would mean looking at the cumulative distribution function\footnote{For a real number $x$, $\mathcal M(x)$ denotes the mantissa of $x$, i.e. the unique real number $\mathcal M(x)\in [0,1)$ such that $x-\mathcal M(x)\in \mathbb Z$.} of 
$$Y_n = \mathcal M ( n U_b +\frac{1}{2} ) -\frac{1}{2} $$
to distinguish the cases $b=0$ and $b=1$. Since when $b=1$, the $Y_n$'s accumulate at 0 while being uniformly distributed when $b=0$, one could test the uniform repartition of $Y_n$'s in  $(-\frac{1}{2} , \frac{1}{2})$ outside of a small centered set.


\subsection{Scatter points}

Principal component analysis allows to compare random data sets : with $X_W(t) = \{ \langle w , t \rangle \}_{w\in W}$, one can study and reduce the covariance matrix
$$X_W(t) X_W(t)^T =\sum_j \lambda_j u_ju_j^t.$$
If $W$ is a BKZ-reduced basis for $\Lambda^\vee$, then 
$$ \lambda_j \sim \lambda_j(\Lambda^\vee).$$

Can we prove something like $\max_j \lambda_j \leq \max_W \|w\|_2^2$?

\subsection{Concentration inequalities}

The following argument shows that the Independance Heuristic of \cite{} is not needed in the study of the total score. The derivation of the bounds for the score is also flawed by the use of the Central Limit theorem, an asymptotic result. An exact control needs an extra computation for the distance between the distribution of the score and the limit normal distribution. This was already pointed out in \cite{}, in which the use of Hoeffding inequality provided exact (and more precise) confidence intervals. We improve on this idea of using exact deviation inequalities by using concentration results for Lipschitz functions, which does not assume independance of the comonent of the score.     

Recall the toy model $X_n = e^{2i\pi nU}$ with $U\sim \mathcal U(0,1)$. If $f : \mathbb R \rightarrow \mathbb C$ is any $1$-periodic function such that $\int_0^1 |f|^2$ is finite, then $f = \sum c_n e_n $ in $L^2$. The random variable $\sum c_n e_n(U)$ converges in $L^2$, and can take any value of $f$. This suggests that a universal bound for arbitrary sums of uncorrelated variables is unlikely. More precisely, if $f(x) = \ sg(x)\sum_k k \chi_{x < k^{-1} } $ when $x\in (-\frac{1}{2},\frac{1}{2})$, and $1$ periodic, then $\mathbb E[f(U)] = 0$ and 
$$\sup_f \frac{1}{N^2}\log \mathbb P(|f(U)| > N) = +\infty .$$ 
One can still control deviations of sums on non-independant variables using the following theorem (a proof can be found in \cite{vershynin2018high}, see theorem 5.1.4 page 99). We denote the vector space $\mathbb R^n$ with euclidean norm $\|\cdot\|_2$ by $\ell^2_n$.

\begin{theorem}
Let $X$ be a random variable distributed either uniformly on the sphere of $\ell^2_n$ of radius $\sqrt n$, or as a centered gaussian vector $\mathcal N_{\ell^2_n}(0, Id_n)$, or as a uniform variable on $(0,1)^n$. Then 
$$ \mathbb  P\left(|f(X) -\mathbb E[f(X)]| \geq t\right) \leq 2\exp{\left(-\frac{ct^2}{\|f\|^2_{Lip}} \right)}$$
\end{theorem}

Let $W\subset \Lambda^\vee$ be a finite subset, and define 
$$f_W(t) = \frac{1}{|W|}\sum_{w\in W} \cos(2\pi \langle w , t \rangle) \quad \forall t \in \ell^2_n.$$
Let us denote by $L$ the positive number $2\pi\max_{w\in W} \|w\|_2$. Then
\[\begin{split}
|f_W(x) -f_W(y)| & \leq \frac{1}{|W|}\sum_{w\in W} |\cos(2\pi \langle w , x \rangle ) -\cos(2\pi \langle w , y \rangle) | \\
		 & \leq \frac{2\pi}{|W|}\sum_{w\in W} \|w\|_2 \|x-y\|_2\\
		 & \leq L\|x-y\|_2\\ 
\end{split}\]
i.e. $f_W$ is a $L$-Lipschitz function.  

If $X$ follows a LWE-distribution, then $\mathcal E[f_W(X)] \sim 1$ (to prove) and the inequality above says that $f_W(X)$ is close to one with high probability.

If $X$ follows a uniform distribution on the fundamental domain, prove that $\mathbb E[f_W(X)]= 0$ and that $f_W(X)$ lies outside of a neighborhood of $0$ with high probability.

\subsection{Bounds for discrete gaussian distributions}

In order to be able to apply the exact bounds for a LWE distinguisher, we prove concentration inequalities valid for discrete gaussians. Recall our notation for the gaussian function 
$$\rho_s(v)=\exp(-\pi \frac{\pi}{s^2}\|v\|^2 )\quad \forall v\in \mathbb R^n.$$
The continuous gaussian measure\footnote{We denoted the Lebsgue measure of dimension $d$ by $m_d$.} is 
$$\gamma_s(A) =  \int_A s^{-n}\rho_s(x)dm_d(x) ,$$
whereas the discrete gaussian on $\Lambda$ is the probability measure 
$$D_{s,\Lambda} = \rho(\Lambda )^{-1} \sum_{v\in \Lambda} \rho_s(v)\delta_v. $$

\begin{proposition}
Let $\xi \sim D_{s,\Lambda}$ and $r > \lambda_d^{(\Lambda)}$, then 
$$\mathbb P(\|\xi \|> r )\leq 2^de^{-\frac{\pi}{s^2} C_\Lambda \tilde r^2}$$
where $C_\Lambda$ is a constant that only depends on the lattice and $\tilde r$ is the closest integer to $\frac{r}{\lambda_d^{(\Lambda)}}$.
\end{proposition} 

\begin{proof}
Let $\{v_k\}$ be a reduced basis for $\Lambda$, i.e. $\|v_k\| = \lambda_k^{(\Lambda)}$. For $a\in \mathbb N^d$, let $\Lambda_a$ denote the sublattice of $\Lambda$ generated by $\{a_kv_k\}$  and let us set 
$$\mathcal V_a = \{t\in E : \|t\|\leq \|t-v\| \forall v\in \Lambda_a\}, $$
a Voronoi cell for $\Lambda_a$. We will need to use, for a vector $t\in E$, the half-space 
$$H_t = \{y \in E : \langle y-t , t\rangle \leq 0  \}$$ 
defined by vectors positively oriented with respect to the median hyperplane between $t$ and the origin. If $\Lambda_{t} =\Lambda\cap H_t$, one has the disjoint union
$$\Lambda = \Lambda_t \coprod \Lambda_{-t} \coprod \Lambda_t^{(0)}.$$
If $u\in \Lambda$, one has
\[\begin{split}
\rho_s(\Lambda_u) & = e^{-\frac{\pi}{s^2} \|u\|^2 }\sum_{v\in \Lambda , \langle v-u ,u\rangle >0 } e^{-\frac{\pi}{s^2} \|v - u \|^2}e^{-\frac{\pi}{s^2} \langle v-u,u\rangle} \\
		& \leq  e^{-\frac{\pi}{s^2} \|u\|^2 } \rho_s(\Lambda) \\ 
\end{split}.\]
This ensures that $\rho_s(\Lambda \backslash \Lambda_u^{(0)}) \leq 2e^{-\frac{\pi}{s^2} \|u\|^2} \rho_s(\Lambda) $. By induction, we get that
$$\rho_s(\Lambda \cap \mathcal V_a^c) \leq 2^d  e^{-\frac{\pi}{s^2} \sum_{k} \|a_ku_k\|^2}\rho_s(\Lambda) ,$$
i.e.
$$\mathbb P(\xi \in \mathcal V_a^c) \leq  2^d e^{-\frac{\pi}{s^2} \sum a_k^2\lambda_k^2} .$$
If $r\geq \lambda_d^{(\Lambda)}$, let $a$ be maximal such that $\mathcal V_a\subset B(0,r)$, thus
$$\mathbb P(\| \xi \| \geq r) \leq  2^d\prod_{k} 2e^{-\frac{\pi}{s^2} a_k^2\lambda_k^2} .$$
\end{proof}

%%%%%%%%%%%%%%%%%%%%%%%%%%%%
\newpage	%%%%%%%%%%%%
%%%%%%%%%%%%%%%%%%%%%%%%%%%%
\section{Lattices}

Every locally compact group admits a left-invariant regular Borel measure, and all of these are proportional. These are called Haar measures. A lattice in a l.c. group is a closed subgroup of finite co-volume\footnote{i.e. the pushed-forward on the quotient of any Haar measure is finite.}. Such groups are always discrete. 

\subsection{Euclidean lattices}
Every lattice of $\mathbb R^n$ is a discrete subgroup, and is of the form 
\[\Lambda_B = \{ Bv \ : \ v\in \mathbb Z^n\} \]
for some invertible matrix $B\in GL(n,\mathbb R)$, called a basis. 

The dual $\Lambda^\vee$ is the euclidean lattice
\[\{ w \in V \ : \  \langle w , \Lambda \rangle \subset \mathbb Z \}.\]
The map 
$$w \in \Lambda^\vee \mapsto \left(t\in V \mapsto \exp(2i\pi \langle w,t \rangle)\right)$$ 
induces an isomorphism 
\[\Lambda^\vee \cong \widehat{V/\Lambda}.\] 
A consequence of which is that every (for instance square-integrable) $\Lambda$-periodic function on $V$ can be expressed as the (hilbertian) weighted sum of characters. This\footnote{I forgot some normalizing constant.} is the isomorphism given by the Fourier transform $\mathcal F : L^2(V/\Lambda ) \rightarrow \ell^2(\Lambda^\vee) $
\[\mathcal F(\varphi)(w) = \int_V \exp(-2i\pi \langle w,t \rangle) \varphi(t)dt. \]
\[\mathcal F^{*}(\psi)(t) = \sum_{w\in\Lambda^\vee} \psi_w \exp(2i\pi \langle w,t \rangle) . \]

\subsection{Poisson formula}
The Fourier transformation 
\[\mathcal F(\varphi ) (\xi ) = \int_V \varphi(v)e^{-2i\pi \xi(x)} dm(v) \quad\forall \xi \in V^\vee.\]
induces an isomorphism of topological vector spaces
\[\mathcal F : \mathcal S(V) \rightarrow \mathcal S(V^\vee)\]
between Schwartz spaces of $V$ and its dual (as a vector space), which extends to an isomorphism
\[\mathcal S'(V) \rightarrow \mathcal S'(V^\vee)\]
of tempered distributions.
The Poisson formula is the statement that for every Schwartz class function $\varphi\in\mathcal S(V)$, 
\[ \varphi (\Lambda + t) = covol(\Lambda)^{-1}(\mathcal F(\varphi)e^{2i\pi \langle \bullet , t\rangle }) (\Lambda^\vee).\]
It follows from the fact that the counting measures on $\Lambda$ and $\Lambda^\vee$ are related by 
\[\mathcal F(\sum_{v\in \Lambda} \delta_{v} ) = covol(\Lambda)^{-1}\sum_{w\in \Lambda^\vee} \delta_w .\]
For the gaussian function $\rho_s(t) = \exp(-s\pi\|t\|^2)$, 
\[\mathcal F(\rho_s)(\xi)= s^{-n/2}\exp(-s^{-1}\pi\|\xi\|^2 ),\]
which yields the classical identity\footnote{$\Theta_{\Lambda}(s) = \rho_s(\Lambda)$.} for thêta functions
\[ \Theta_{\Lambda}(s) = covol(\Lambda)^{-1}s^{-n/2}\Theta_{\Lambda^\vee}(s^{-1}).\]

\subsection{Random euclidean lattices}
The map that sends a basis to the generated lattice yields a measurable model 
\[\mathcal L_n \cong  GL(n,\mathbb R) / GL(n,\mathbb Z) \]
for the space of lattices. Since $GL(n,\mathbb Z)$ is a lattice in $GL(n,\mathbb R)$, $\mathcal L_n$ admits a unique invariant probability measure $\mu_n$. In a similar manner, lattices of covolume $\rho$ admit an invariant probability measure from the model 
\[\mathcal L_{n,\rho} \cong \mathcal L_{n,1}\cong SL(n,\mathbb R) / SL(n,\mathbb Z)\]
These measures are called the Haar-Siegel probability measures in the litterature.

% Ajouter les résultats sur l'heuristic gaussienne vu comme une approximation de E[f(L)]  = \int f d\mu [section 2.5 de Dadush Regev]
% d'où E[ | L\{0} \cap A|] = Vol(A)
% Goldstein Mayer 2003 : la mesure uniforme (en fait ils considèrent la mesure uniforme sur l'ensemble fini des réseaux entiers de covolume fixé égal à N) sur les réseaux de congruence converge en un sens fort vers la mesure \mu . Les réseaux de congruence ne sont pas exactements ceux là, leur volume est un diviseur de q^n, de plus un réseau entier de covolume N ne contient pas nécessairement NZZ^n.
% Pouly Shen donne dans D3 et D4 une version quantitative de l'heuristique gaussienne pour les réseaux de congruence. 

\subsection{Congruence lattices}
A congruence lattice\footnote{The accepted terminology in the computer science community is $q$-ary.} is a lattice $\Lambda$ of $\mathbb R^n$ such that 
\[q\mathbb Z^n < \Lambda < \mathbb Z^n,\]
for some positive integer $q$, called the modulus. When useful, we will use the term $q$-congruence lattice if we want to make the modulus explicit.
 
As inclusion of lattices $\Lambda_{B_0} < \Lambda_{B_1}$ translates as a relation
\[\exists g \in  GL(n,\mathbb R) \cap M_n(\mathbb Z) , B_0 = B_1g , \] 
these inclusions entails basis of congruence lattices are matrices in 
\[\{B\in M(n,\mathbb Z) : \exists g\in GL(n,\mathbb Z[\frac{1}{q}] ) \ : \ Bg = gB = qI_n \}\]
%$M(n,q\mathbb Z)\cap GL(n,\mathbb R) \subset GL(n,\mathbb Z[\frac{1}{q}])$, i.e.
%\[\mathcal L_{n,q} \cong \left(M(n,q\mathbb Z)\cap GL(n,\mathbb Z[ \frac{1}{q} ])\right) / GL(n,\mathbb Z)\] 
Let $\Gamma_0 , \Gamma_1$ be two lattices in a l.c. group $G$ such that $\Gamma_0$ is of finite index in $\Gamma_1$, then 
\[ \text{covol } \Gamma_0  = [\Gamma_1 : \Gamma_0 ] \ \text{covol } \Gamma_1 .\]
As a consequence, $\text{covol }\Lambda $ is equal to the index $[\mathbb Z^n : \Lambda]$ and divides $q^n$.

Let $\Lambda$ be an euclidean lattice, and $F$ be a finite group of cardinal $k$. Let us define $\mathcal L_{n , F}(\Lambda ) $ be the set of lattices $\Lambda'$ containing $\Lambda$ and satifying that $\Lambda' / \Lambda $ is isomorphic to $F$. This set is finite and it follows from \cite{ordentlich2022new} (proposition 2.1) that
\[\int \varphi d\mu_{n,k^{-1}} = \int_{\mathcal L_n} \frac{1}{k}\sum_{\Lambda'\in \mathcal L_{n , F}(\Lambda )} \varphi(\Lambda') d\mu_{n}(\Lambda)  \quad \forall \varphi \in C_c(\mathcal L_{n,k^{-1}} ) \]
As a consequence, the uniform probability measure on $q$-congruence lattices is the law $\Lambda'$ obtained by drawing $\Lambda$ according to $\mu_{n,1}$, and then drawing $\Lambda'$ in $\mathcal L_{n,\mathbb Z^n /q\mathbb Z^n}(\Lambda)$ uniformly, conditionally to the event $\{\Lambda = q\mathbb Z^n\}$.   
%%%%%%%%%%%%%%%%%%%%%%%%%%%%%%%%%
\section{Relative BDD}%%%%%%%%%%%
%%%%%%%%%%%%%%%%%%%%%%%%%%%%%%%%%

Let $G$ be a locally compact abelian group endowed with a metric $d : G\times G \rightarrow \mathbb R$, $H$ a finitely generated subgroup and $\mu$ a probability measure on $G/H$. Define the following problem
$$\mathcal P (G,H,\mu) : \text{Given } x \text{ s.t. }x+H \sim \mu , \text{ find argmin}_{a\in H} d(x,a).$$

In particular instances, this problem specializes to well studied problems. The generalization is motivated to highlight some subtle points : it can happen that one has to deal with codes while starting with a problem related to lattices. For example, this phenomenon arises in LWE because the quotient of the LWE lattice $\Lambda_q(A)$ by $q\mathbb Z^n$ is not a lattice but a finite group. 
\[\begin{array}{|c|c|c| c | c|} 
\hline
G             & H                          & d     &  \mu              & \mathcal P (G,H,\mu)\\
\hline
\mathbb R^n   & \Lambda                    & \ell^2 & \chi_e            & \text{CVP}     \\
\mathbb F_q^n & \mathcal C                 & d_H    & \text{uniform on fixed weight}    & \text{Decoding}\\
\mathbb Z^{\text{nrow}(A)}  & \Lambda_q(A) & \ell^2 &  A \chi_s +\chi_e & \text{LWE}     \\
\hline
\end{array}\]

Our work focuses on breaking $BDD(\Lambda)$ into a series of problems with the hope of decreasing the complexity :
\begin{itemize}
\item[$\bullet$] find $\Lambda_0$ such that $BDD(\Lambda_0)$ has quasi-polynomial complexity\footnote{Or just at least a better complexity than expected $BDD$ complexity.}.
\item[$\bullet$] apply the dual attack to reduce $BDD(\Lambda)$ to $BDD(\Lambda_0)$. 
%\item[$\bullet$] for $G = \Lambda/\Lambda_0$, $BDD(\Lambda)$, find $b\in B$ such that  is equivalent to 
\end{itemize} 

We also provide a rigorous statistical model for the attack. For the distinguishing step, we establish an exact deviation bound for the score which does not suppose independance on the terms of the sum. We explicitely link the maximization of the score function to the optimization of the p-value of the tests. The end of the work is devoted to a discussion of the parameters for which the attack is efficient, and comparison with previous work. 

Our result allows a modification of the attack as described in \cite{pouly2023provable}. As we don't need the independance of the summands in the distinguisher, we can use a sieve to generate our sample of short vectors $W$ in $\Lambda^\vee$. This is more efficient than using independant calls to a gaussian sampler iteratively. Moreover, it is not clear that the results in \cite{pouly2023provable} apply even if $W$ is sampled with $|W|$ calls to a sampler, since the different terms are not independant once the randomness of the target is taken into account\footnote{if $w,w'$ are independant and $t\sim D_\Lambda$, $\langle w , t \rangle $ and $\langle w' , t \rangle $ are not necesserily independant.}.   
 
The distinguishing step depends on a sublattice $\Lambda_0 < \Lambda$. One builds a finite set of short vectors $W\subset \Lambda_0^\vee$ and define the distinguisher
$$f_W(t) = \frac{1}{|W|} \sum_{w\in W} \cos (2\pi \langle w, t\rangle ).$$ 
This function is $\Lambda_0$-periodic and thus 
$$ t_0^* = \text{argmax}_{t\in V} f_W(t) = \text{argmax}_{t\in V/\Lambda_0} f_W(t+ \Lambda_0) $$
satisfies $t - t_0^*$ follows a BDD-distribution on $\Lambda_0$.

In the setting of LWE lattices, let us write $A  = [A_{dual} | A_{guess} ]$. With the notations as above, 
$$\Lambda_0 = \Lambda_{q}(A_{dual}) < \Lambda_q(A).$$ 
If the columns of $A$ are independant, we have that 
$$\Lambda_{q}(A) / \Lambda_{q}(A_{dual}) \cong \Lambda_{q}(A_{guess})$$ 
and when the attack is succesful, i.e. $s_{guess} = s_{guess}^*$, we get $t-As_{guess}^* = A_{dual}s_{dual} + e$.

In the setting where $\Lambda / \Lambda_0 = G$ is a finite abelian group (as in \cite{ducas2023does}), we get the problem $\mathcal P(G,\mu)$. In \cite{ducas2023does}, the optimisation of $f_W$ is done first with computing its Fourier transform, which for particular case of groups such as $(\mathbb Z / 2 \mathbb Z)^n$ can be done very efficiently. 

\subsection{Concentration inequalities for lipschitz functions of discrete gaussians}
In order to be able to apply the exact bounds for a LWE distinguisher, we prove concentration inequalities valid for discrete gaussians. 

\begin{lemma}
Let $X$ be a real random variable and $\varphi : \mathbb R \rightarrow \mathbb R$ be a differentiable function such that $t\mapsto \varphi'(t)\mathbb P(X>t)$ is integrable. The following formula holds
$$\mathbb E [\varphi(X)] = \varphi(a) + \int_{a}^{+\infty} \varphi'(t)\mathbb P(X > t) dt \quad \forall a \in \mathbb R.$$
If $\lim_{t\rightarrow -\infty}\varphi(t) =0$, the following also holds
$$\mathbb E [\varphi(X)] = \int_{-\infty}^{+\infty} \varphi'(t)\mathbb P(X > t) dt.$$
\end{lemma}

\begin{proof}
Applying the fundamental theorem of calculus,
$$\varphi(x)-\varphi(a) = \int_a^x\varphi'(t)dt$$
Evaluating the expectation and Fubini interversion theorem concludes the first part.
For the second part, we let $a$ tend to $-\infty$ to get :
$$\varphi(X) = \int_{-\infty}^X\varphi'(t)dt = \int_{-\infty}^{+\infty} \varphi'(t) 1_{\{X > t\}} dt.$$
and concludes in the same manner.
\end{proof}

We call a real random variable $X$ sub-gaussian if there exist constants $C , a>0$ such that 
$$\mathbb P( X > t ) \leq Ce^{-at^2} \quad \forall t >0.$$ 
If we want to keep track of the constants involved, we will call such a variable a $(C,a)$-sub-gaussian random variable.

\begin{theorem}
Let $(M,d)$ be a metric space, and $\xi$ a random variable with values in $M$, such that for every (or equivalently for one) point $m\in M$, the random variable $d(\xi, m)$ is $(C,a)$-sub-gaussian. Let $f : X\rightarrow \mathbb R$ be a lipschitz function with lipschitz constant bounded by $L>0$. Then  
$$\mathbb P(|f(\xi)  - \mathbb E[f(\xi)]| > t) \leq   \frac{C^2a^3}{\pi L^2} \cdot t^2 e^{-\frac{ L^2}{8a}t^2} .$$
\end{theorem}

\begin{proof}
Let $\tilde\xi$ an independant copy of $\xi$, $m\in M$ and $\lambda>0$. We have 
\[\begin{split}
\mathbb P(|f(\xi)  - \mathbb E[f(\xi)]| > t) & \leq e^{-\lambda t} \mathbb E[e^{\lambda |f(\xi)  - \mathbb E[f(\xi)]| } ] \\ 
		& \leq  e^{-\lambda t} \mathbb E[\exp({\lambda \mathbb E[|f(\xi)-f(\tilde \xi)| \ | \ \xi] })] \\
		& \leq  e^{-\lambda t} \mathbb E[e^{\lambda |f(\xi)-f(\tilde \xi)| }] \\
		& \leq  e^{-\lambda t} \mathbb E[e^{\lambda Ld(\xi , \tilde \xi) }] \\
		& \leq e^{-\lambda t} \mathbb E[e^{\lambda Ld(\xi , m) }]^2 \\
\end{split}
\]
The preceeding lemma applied to $\varphi(u) = e^{L\lambda u}$ and $X=d(\xi,m)$ ensures that
\[\begin{split}
\mathbb E [e^{\lambda L d(\xi , m)} ] & = \lambda L \int_{-\infty}^{+\infty} e^{\lambda L u} \mathbb P(d(\xi,m) > u) du \\
	& \leq C\lambda L \int_{-\infty}^{+\infty} e^{\lambda L u-au^2} du \\
	& \leq C\lambda L e^{a\frac{\lambda^2L^2}{4a^2}}\int_{-\infty}^{+\infty} e^{-a(u-\frac{\lambda L}{2a} )^2} du \\
	& \leq C\lambda L \sqrt{\frac{a}{\pi}} e^{\frac{\lambda^2L^2}{4a}} .\\
\end{split}\]
Replacing in the last inequality, we get
\[\begin{split}
\mathbb P(|f(\xi)  - \mathbb E[f(\xi)]| > t) & \leq C^2 L^2 \frac{a}{\pi} \cdot \lambda^2 e^{\frac{ L^2}{2a}\lambda^2 -\lambda t} \\ 
\end{split}
\]
Setting $\lambda = \frac{at}{L^2} $, we get the result. I did not dare minimizing in $\lambda$. 
\end{proof}

Suppose now that there exist constants $C , a , t_0>0$ such that 
$$\mathbb P( X > t ) \leq Ce^{-at^2} \quad \forall t >t_0.$$
The same method as in the previous result yields 
$$\mathbb E [e^{\lambda L d(\xi , m)} ] \leq e^{\lambda L t_0}  + C\lambda L \sqrt{\frac{a}{\pi}} e^{\frac{\lambda^2L^2}{4a}} $$
hence 
$$\mathbb P(|f(\xi)  - \mathbb E[f(\xi)]| > t) \leq e^{-\lambda t} \left(e^{\lambda L t_0}  + C\lambda L \sqrt{\frac{a}{\pi}} e^{\frac{\lambda^2L^2}{4a}}\right)^2 $$
If $\lambda = \frac{at}{L^2} $, we get
\[\begin{split}
\mathbb P(|f(\xi)  - \mathbb E[f(\xi)]| > t) & \leq e^{-\lambda t} \left(e^{\lambda L t_0}  + C\lambda L \sqrt{\frac{a}{\pi}} e^{\frac{\lambda^2L^2}{4a}}\right)^2 \\
					& \leq \left( e^{at_0^2}e^{-a(\frac{t}{L} - t_0)^2} + \frac{C^2a^3}{\pi L^2} \cdot t^2 e^{-\frac{ L^2}{8a}t^2} \right)^2 \\
\end{split}
\]

Let us set our notation for the gaussian function of parameter $s>0$, 
$$\rho_s(v)=\exp(-\pi\frac{\|v\|^2 }{s^2} )\quad \forall v\in \mathbb R^n.$$
The continuous gaussian measure\footnote{We denoted the Lebsgue measure of dimension $d$ by $m_d$.} is 
$$\gamma_s(A) =  \int_A s^{-n}\rho_s(x)dm_d(x) ,$$
whereas the discrete gaussian on $\Lambda$ is the probability measure 
$$D_{\Lambda , s} = \rho_s(\Lambda )^{-1} \sum_{v\in \Lambda} \rho_s(v)\delta_v. $$
Recall the following corollary of a theorem of Banaszcyk theorem (lemma 1.5 in \cite{banaszczyk1993new}).

\begin{theorem}
Let $\Lambda$ an euclidean lattice of rank $d$ and $\xi \sim \mathcal D_{\Lambda , s}$,
$$\mathbb P(\|\xi - v\|> r ) \leq \exp(-\pi (\frac{r}{s} -\sqrt{\frac{d}{2\pi}})^2) \quad \forall r >\sqrt{\frac{n}{2\pi}}s $$
\end{theorem}

Let $W\subset \Lambda^\vee$ be a finite subset, and define 
$$f_W(t) = \frac{1}{|W|}\sum_{w\in W} \cos(2\pi \langle w , t \rangle) \quad \forall t \in \ell^2_n.$$
Let us denote by $L$ the positive number $2\pi\max_{w\in W} \|w\|_2$. Then
\[\begin{split}
|f_W(x) -f_W(y)| & \leq \frac{1}{|W|}\sum_{w\in W} |\cos(2\pi \langle w , x \rangle ) -\cos(2\pi \langle w , y \rangle) | \\
		 & \leq \frac{2\pi}{|W|}\sum_{w\in W} \|w\|_2 \|x-y\|_2\\
		 & \leq L\|x-y\|_2\\ 
\end{split}\]
i.e. $f_W$ is a $L$-Lipschitz function.  

If one uses a sieve to sample vectors from $W$, we can give the following bounds 
$$2\pi gh(\Lambda) \leq L \leq 2\pi\sqrt{\frac{4}{3}}gh(\Lambda) $$

%%%%%%
\section{Best score and p-value}

For $c >0 $, let us denote 
$$\mathcal R_{t,c} = \{f_W(t) > c\}$$
a rejection region defining a statistical test. Choosing $t$ that maximizes $f_W(t)$ is equivalent to selecting the statistical test with minimal p-value. Indeed, let $t_0 = \text{argmax }f_W(t)$ and $\alpha^*_t$ be the p-value of the test associated with $\mathcal R_{t,c}$, and $c^* = f_W(t_0)$. Then 
$$\alpha^*_{t_0} \leq  P(\mathcal R_{t_0,c^*} ) \leq \alpha^*_t \quad \forall t\neq t_0.$$

\section{Questions} 

Why does Pouly-Shen's paper have no restriction on the dimension of the lattice ?

Can we find a set of parameters in the contradictory regime of Ducas-Pulles with an attack that works anyway ?

The paper of Carrier et al. gives an analytical derivation of the curves of Ducas-Pulles. Are we sure these curves explained the 'in-vitro experiment' of DP without explaining anything about the attack?

%%%%%%%%%%%%%%%%%%%%%%%%%
%%%%%%%%%%%%%%%%%%%%%%%%%
\bibliographystyle{plain}
\bibliography{biblio}
\end{document}































